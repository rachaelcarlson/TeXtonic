% Created 2017-05-19 Fri 12:31
\documentclass[11pt]{article}
\usepackage{fontspec}
\setmainfont{Times New Roman MT Std}
\usepackage{graphicx}
\usepackage{longtable}
\usepackage{float}
\usepackage{wrapfig}
\usepackage{rotating}
\usepackage[normalem]{ulem}
\usepackage{amsmath}
\usepackage{textcomp}
\usepackage{marvosym}
\usepackage{wasysym}
\usepackage{amssymb}
\usepackage{hyperref}
\tolerance=1000
\author{Rachael Carlson}
\date{Conducted May 11, 2017}
\title{Michael Millham Interview}
\setlength{\parskip}{1em}
\hypersetup{
  pdfkeywords={},
  pdfsubject={},
  pdfcreator={Emacs 25.1.1 (Org mode 8.2.10)}}
\begin{document}

\maketitle
\flushleft

\emph{I have attempted to omit abandoned sentences and filler-words where appropriate in order to ensure that this document is as useful as possible. I have also written words without dialect.}

\emph{M: Michael Millham; J: John Stropes; R: Rachael Carlson; K: Keleren Millham}

(00:26)

J: wonderful! Very nice to see you. Rachael Carlson here.

R: Hello. Nice to meet you.

J: Michael, I know that we have a lot to catch up on. Didn't we talk about 20 or 30 years ago. 

M: Yeah, I think that's about right.

J: Yeah. How have you been in the mean time?

M: Playing guitar and well.

J: Good

M: And plenty of it.

J: I still run into Dan Schwartz. In fact, he was just sending me some text messages this morning.

M: Dan is a long time friend, for sure.  

J: He's a good guy. We've got a guitar builder, Jason Kostal, who's coming into town. He just build a guitar for Willie Porter apparently. And Dan's been going to be coming down and have dinner at the end of the month. So I'll be seeing him soon. He was just on his way to his gig at the airport.

M: That's pretty fun. Will Boulé ran into him at the airport the first time he came out to work with you, I believe.

J: So, the. Thanks for sending Will this way, he has been a real treat. What a load of enthusiasm.

M: Real quick. The story on Dan Schwartz is that my wife and I used to tour with him when he would come out this way. So we have done a bunch a trips up and down the California coast and Oregon and things like that. Playing colleges and coffee houses and pubs and whatnot. Now we met through Billy McLaughlin who I met when I was probably younger than you. 

J: And I met Dan through Tuck Andress initially. I remember Tuck had run into him and he was in for a concert. And he said ``Hey, this kid will be in touch with you. Treat him okay.''

M: Cool. But I think it panned out already. He got some good skills out of it. Some standard notation and everything. 

J: So everything seems to lead forward. Right now Rachael is doing a paper on—it's kind of a survey on what guitar programs around the country and around the world include finger-style guitar and how they manage to do that. I'm talking about college-level programs. So I thought it'd be interesting to interview you on this topic. There aren't that many to speak of. I mean are aware of any other programs which include finger-style guitar in anyway whatsoever.

M: Aside from yours, not really. Not really at all. It's largely a function of having a player that is excited enough about the genre to push for it. Is what it boils down to. 

J: I know there are a few cases where people are teaching guitar in colleges and because they have some interest and expertise some of the students wind up studying finger-style guitar. And there's the program in Denver that Alex de Grassi always reminds me of with Sean McGowan. 

M: Oh yes.

J: Who you could say that's kind of the finger-style jazz program because Sean plays with his fingers but its strikes me as pretty much being a jazz program where finger-style guitar is used. But anyway, let's start from the beginning. Let me ask you a couple background questions.

M: Shoot.

(4:00)

J: What colleges or universities do you currently teach at. 

M: I currently teach at Gonzaga University in Spokane Washington. And then Eastern Washington University which is a state school. That's where Will came from. That's in Cheney, Washington. Which geographically—I mean it is 18 miles away.

J: So when did you begin at each of those schools?

M: I just finished, literally, I turn in grades this week at Gonzaga, and that will be my 16th year there. Eastern is 10. 

J: When you began at each of these schools, in what capacity were you working? I mean, what you were doing on the one hand and what programs you were involved in. 

M: They're slightly different. Gonzaga is out of the Jesuit University system. And for their music, they do have music majors, and they have some that are really good.  But its largely a, their idea is to have well-rounded graduates. So they love a lot of people participating in music. In our heyday we would have a hundred people taking guitar with three instructors. And I got hired originally to teach beginning guitar classes. Literally, we were on our way out of the Bay area to Portland—we were doing some gigs in our van (the same one that Dan Schwartz has been). And the cell phone kicked on. You know, the service kicked on coming out of the Red Woods and it was Gonzaga saying can you come back and teach these classes in a couple weeks. Because I was one of the better known guitarists in Spokane at that time. Do you mind doing this? And I was like ``yeah, that sounds like fun.'' So that's how I ended up at Gonzaga. And then of course, you get people that come along and want to major in guitar and get that kind of thing happening. My first guy that ever graduated from Gonzaga a kid named Evan Everest who was the first person to graduate on the guitar performance track—classical. And his now wife but then fiance was a journalism major and she got a job out in the middle of nowhere because that's what happens when you get a journalism degree.  You've got to take the city beat your first job. And she ended up in Ukiah, California. And guitar being a small thing, that's where Evan met Alex de Grassi. So Evan, the first guy to graduate with a performance degree from Gonzaga is the one who introduced me to Alex. And he was the last of finger-style guys, my heroes, who I had not met. I had met all of them. You know, I met Pierre, Michael, and Laurence you know. Alex was the last dude. Great guy.

(6:30)

J: So you were hired initially at Gonzaga to teach class guitar?

M: Yep, beginning class guitar and then you know private lessons as well. Overflow. Through those I ended up working with majors. And that's really. The beginning class is more recreational. It's people who are pre-med and mechanical engineering and life sciences and they are taking guitar because cool credit. And it's like ``As long as you're taking guitar, let's make sure that you're good at it.''

(7:00)

J: So they have a performance degree on the books when they hired you?

M: The performance degree came shortly after I was hired. They had an overarching bachelor of arts degree. That's why Evan was the first performance degree, he's not the first guitarist to come through there. But he is the first to do the Junior Senior recital for thirty minutes one hour track. 

J: Was that a BFA then?

M: That would be a BM.

J: BM okay. And do you they also have a, yeah okay. And the class guitar was just an elective credit for whoever wanted to take it. Was there other guitar instructors there at that time? 

M: Yes. Yes. They actually—and that's the deal, they had two instructors, who for various reasons, departed from the program leaving a void. So the guy that was left and everybody who was outgoing said ``well, the best guy to take this, as the quote goes, Michael Millham, and good luck finding him.'' Because we were vagabonds we lived in our van essentially. All of our stuff was storage and we didn't have a lease or an apartment. We had cell phones and a Toyota Previa. And about four grand worth of PA. We had these speakers and a guitar or two. And so that's what the deal was with the cellphone.

(8:07)

Now Eastern was a different deal. That thing is, I actually left another position with a local university to start the program from scratch there. And kind of jump it. I started with one student. That's a music major. It's a state school, a little bit more affordable. I only work with Majors there. I don't teach anybody recreationally. They have performance majors. But music majors, music ed, composition and guitar performance.  

J: So those are the three programs at Northeastern that include guitar? Music ed, what did you say, composition and guitar performance?

M: Those are the tracks where I tend to have a lot guitarists. I've taught some people that are saxophone performance and guitar is a second instrument. It's largely either instrumental performance. The large majority something like seventy percent are music ed, getting ready to go out and teach in the secondary ed market. You need a primary instrument and some percentage choose guitar. And in composition there's a lot of people that do comp track that are interested in writing for guitar as well.

(9:11)

J: Its funny by comparison to our program here, the licensure requirements for music ed in the state are so heavy that its all a person can do to get through the program without having to without allowing themselves time to take guitar. Which would be very useful in a school setting. So I'm glad to hear that you have a little more flexibility where you are. 

M: So your music ed majors have to have a primary instrument though.

J: Yeah but it's like music ed band, music orchestra, music ed vocal, and music ed guitar is not in the list. 

M: That is astonishing. As you probably know, the NAMM association has told us that guitars have out sold every other instrument combined since like 1974 or something. People that come out with guitar proficiency, that can actually play, like sit down in front of people and play a piece, they can pretty much write their own ticket and start a class guitar thing and turn a .6 appointment into a full-time appointment just by that. So by that, at least in Washington its a great time to do guitar. 

J: That's nice to hear. So take your student load at each of these schools can you give the proportion of one type of student, one program type of student to another. 

M: As far as like—

(10:45)

J: Like at Northeastern, how many students do you have and what programs are they coming from? Are they participating in.

M: Its split you know, I would say the majority of people that I have at Eastern Washington are performance majors. They're there to play guitar. They're interested in going on and working professionally in some capacity. Second to that would be composition mainly because we have a brilliant comp guy. Who is sort of cutting edge. He was just recently written up in \emph{Newsweek} and \emph{The New York Times} for some of his ideas. So attracts people—

(11:30)

J: So what is this composer's strength. What instruments does he enjoy working with? Writing for?

M: Experimental music and computer music. And he is on the cutting edge for working with natural sciences and composition. One his students is a comp major whose primary instrument is guitar just wrote this great piece of music based on the proteins of the heart and how the protein sequences unfold. I'm not a science guy. I don't profess to understand it but I was watching the premiere of that at a recital the other day and thinking ``this is the first new thing I've seen in music since maybe minimalism. Very cool.'' 

J: So was it a guitar tune.

M: Yes. A three movement suite based on the proteins of the heart.

J: Oh, that's nice. I mean its nice that your composition instructor's happy to include guitar as instrument and not deal entirely with computer generated music. 

M: To be honest, a lot of this program stuff is really just personality driven. If I take a comp student, I'd be like let's look at this Carcassi etude. And what I want to you do this week, you need to write something to get your hands, you know you can either do the Carcassi etude, which is great, or we can take the same basic chord progression, analyze it, and then go okay, take another meter, put it in seven and extend them all by using jazz harmonies. One 7ths and 9ths or whatever and come back with something. Once the comp guy figured out that these tunes were actually going to work he sort of gave me free rein to include them. 

(13:00)

He's got his program but he loves the addition of guitar. Its been a nice collaboration. That's because he just a great guy and I like to think that I get along well with people. That's how it worked out.

J: That in itself would an important component of any university program. Folks who get along and can create a friendly atmosphere for growth. 

M: It boils down to trust. We don't have a finger-style program at Eastern. But what we do have the ability to do is do a BA in performance. We have some flexibility. And the performance degree Bachelor of Music is ``here is your literature'' and you're expected to have movement from a 20th century concerto by your senior year and you have to have your Baroque suite. All the stuff that goes with classical music, René Izquierdo stuff. And then the BA is like ``well, you can sort of do whatever you want.'' What I'd like is for someone to do a junior and senior recital working on the music of Alex de Grassi and Michael Hedges and Pierre Bensusan if they so desire. Or jazz or whatever it is. So that ends up being the catch-all. And once there's a certain amount of trust. Will Boulé wanted to pursue that and its like ``well, let's pursue it.'' 

J: Now what was Will's degree? The name of it?

M: BA in guitar performance with an emphasis on finger-style guitar. If he wanted pursue guitar performance he would have had to learn lute suites and Tarrega and things like that. I think that's great if you enjoy that music and you live for it. And you wake up hungry to just like dissect it. But if your thing is Pierre Bensusan and Laurence Jubar arrangements and all that then I wouldn't wish another for you. Like I wouldn't want to study heavy metal myself even though I like listening to it. Everybody has their own thing. For Will, let's make this degree work for the student rather than cram the student into a little degree box/basket. There is enough trust in the department. The first time Will got up and played a good rendition of ``Layover'' everybody's like ``this things cool, let's run with it.'' 

(15:00)

J: Let me go back I want to make sure that I understand this clearly. Will's degree was a BA in Music. 

M: Yeah.

J: And you're calling that a performance degree.

M: It was a BA in Music, Emphasis: Guitar. We are using that secondary holder as a catch-all for people who don't want to do the lute suites.

J: And the people who want to do the lute suite. What degree are they pursueing?

M: Bachelor of Music Performance.

J: Okay. So you consider the BA to be a looser structure where more is possible? 

M: It's a great degree. All it boils down to is a little bit more liberal arts. Totally legit degree and we are using that as a way to allow students to participate in what they want to do.

J: Yeah, more flexibility. 

M: We can have Mark Hyphen from the San Francisco Conservatory, Michael Partington from the UW come and do the classical recitals but we can also bring up Pierre Bensusan, Laurence Jubar, Alex de Grassi. All of whom have come to visit the program.

J: That's nice. You know the way you put it reminds me, I don't know if you remember who Duke Miller was at the University of Southern California in Los Angeles where Billy McLaughlin went to school.

M: Yep.

J: With Scott Tennant and Bill Kanengeiser. It was a good class. They just had programs in classical guitar, and studio guitar was the way they divided it up. Maybe they still do. Duke allowed Billy to exist. Knowing that he wasn't aiming at either of those. And it was really nice. It was nice that a person like Billy has been allowed to exist. Without the good judgment and generosity of that program director at that time, he would have had to fit into a square box. 

M: A degree is just an abstract concept until somebody gets inside it and vivifies it. It's like Aaron Shearer letting Michael Hedges exist in Peabody. To some extent that happened. If that guy was forced to study 20th century concertos and lute suites, he probably would have been good at it, but the musical landscape of guitar would look vastly different. I'm sure Billy could have done ``Capricho Arabe'' just as well as anybody but we wouldn't have somebody with an absolutely identifiable signature sound. I truly believe that we are here to support the students, in my opinion. 

J: You know the story with Michael though. He was actually scrubbed out of the classical guitar program after his first year at Peabody. 

M: Yeah. I think he was working with Ray Chester or something. 

J: Yep. Again, maybe its just another way of looking at what turned out for the best for Michael. The fact that he wasn't accepted into this more tightly delineated program.

M: In a more kindly way, wasn't Astor Piazzola scrubbed out of the comp program with Nadia Boulanger. That whole quote about ``you can either be a second-rate classical composer or go back and be the best composer for Argentinian tango.''

J: That's right. 

M: Take Will for example, he could have done the classical thing. He's a very smart guy. He's a wicked hard worker but it's like ``why, he likes this other thing and he's good at it. We need more of this other thing than we need people doing the 20 Sor studies.'' I think that there's more opportunity for graduates. The finger-style thing is not going away. And there's more opportunity for people who go out and start finger-style programs in community college than there is classical at this point. 

J: How would you imagine, if someone like Rachael is just coming out of UWM with the second masters degree, one in Liberal Studies and one in Finger-style guitar performance. How would you recommend that someone like Rachael would go out and try to pitch the idea of a finger-style guitar program at on of hundreds of schools in the US?

(19:17)

M: Liberal studies means you can be a blogger. You are talking about somebody who will be carving their own niche anyways. I will say this there's this idea that you are going to go out with your degree and your going to get tenure-track position and all that. And those things are all kind of going by the wayside. At the time that I have been at Eastern I have seen the amount of tenure—I've been there 10 years, right?—and the number of tenure-track positions have been cut in half in the time that I have been there. Everything is going to a part-time deal. And I'll tell you the whole reason that I'm teaching at these universities and not somebody else is because my picture was in the Weekly and I was on every TV station. They'd play a radio ad and I had CDs and stuff up. If somebody can play well, and they get well known in a town that does not have a guitar program \emph{and} they have the piece of paper, then pitching the [unintelligible] guitar program. And then, once you get in you can teach what you want. Why not teach a finger-style program? It sounds cool. 

J: It sounds like you're saying too that once you get in, you have a lot of flexibility to move a program into the area of your own specialty. 

M: One thing that everybody needs to remember, in education we are moving to a business model nationwide. Make no mistake about it. It's no different than pitching yourself to a theater for putting on a guitar summit. Pitching yourself to the weekly publication in your town for an article. Pitching yourself to a record label, if those even exist anymore. I guess some are still out there. The whole point is that you have to create your own deal. A lot of these is, a piece of paper is a gateway so when they are doing their accreditation they can say ``Yeah, this person has a masters. And yeah, its from Wisconsin's finger-style guitar, but it's still a Masters in Music.'' And then its like, do students like you? Are you producing students in good numbers? And are they producing good sounds? And that is what it boils down to. This idea that a degree will buy you a job is false. You're skill and your attributes buy you a job and your degree is an envelope to package it in, in my opinion. 

J: You mentioned living in your van with cell phone. Was this your early idea for a path to drum up your own place in the world?

M: No. I'm married to a singer and we put a bunch of tunes together. The very first performance that I did that wasn't classical—when we're out of school we moved to Spokane and I got a chance to do an opening for Ed Gerhard. And the guy that was producing went ``Hey! You and Jill are doing some stuff. We'd love to have you open for Ed Gerhard.'' At the time, we had written four tunes that were really, really good. Strong finger-style. And we're like ``Yeah, maybe. I don't know. What are you looking for?'' And he goes ``Maybe four tunes. Just a short set.'' And I go ``You're on!'' 

(22:15)

And the response was good. Finger-style and voice. Big magnetic pickup and all that kind of stuff. Well let's do more of this. We like performing. I had to do a CV recently to—you have to do these things to keep your employment—and I was going to through all our taxes, which fortunately I've kept, and I've discovered that we have averaged over a hundred shows per year for over 20 years. We just never stopped. And back when we were touring with Dan Schwartz it was more like 150 a year because we are playing the rinkeedink little coffee houses on the way in between colleges for 50 bucks. Muddy Waters in Arcadia California for 8 people and those things. We still play out all the time because its what we want to do. So it wasn't like it was a grand plan and we're going to conquer the world with arts. No, it's fun. We're young and we can live in a van. And we're together let's go play a bunch of shows. And we just haven't stopped.

(23:05)

Got one this weekend. Two. Three. Three this weekend.

J: Sounds good. Rachael? You have some questions?

R: You've pretty much asked most of the questions that I had. I get the impression of how you're going to answer this question already but I'll let you know what the question is. I'm wondering what your thoughts are on the presence of finger-style guitar in a university. Do you think it belongs there? Should it be there? Or would it be better for the health of finger-style guitar as a genre to not be in a university setting?

M: Yeah. That's a can of worms. I think its a great idea because more options is better for degree programs. Again, my opinion. 

R: Yeah.

M: The push on—if you look at the history—you know John going to be able to—you've probably forgotten about more programs than I know. You've been in the game a long time and started a bunch of programs.  But my sense is this guitar thing kind of, you know, sort of is a byproduct of maybe 1928 and Andres Segovia coming over and maybe getting people to start to think a little bit differently about guitar versus banjo or whatever. Then you start getting some of the first programs, early on. On the coast, big cities. And they became pretty much a classical thing. We're starting to get where there's a jazz thing. And now the big push is to do what's called modern band. You've probably heard that. Popular music programs. I don't know. Is the program that you're working with an accredited music school? 

R: Here?

M: Yeah.

R: Yes. 

M: Like through NASM?

J: Yeah. We're at the University of Wisconsin-Milwaukee. So its a state university. 

M: Then you know that you maintain accreditation you need to have people who have doctorates in music education. That's the most recent. You may have had to do a search if you didn't have somebody already. To maintain accreditation through NASM you need to have somebody who has a doctorate in Music Education. A lot of those doctorates are coming out of coastal schools more than I can think of in particular. And there's a real push with that to do popular music. There's actual programs where you can go study Dave Matthews and Pat Metheny and things that are popular. And when you look in that context, and you spread out in that direction, then why not have something where we're pushing towards proficiency with Pierre Bensusan, and Alex de Grassi and John\ldots{}whoever? 

(25:42)

J: Are you saying that programs that include popular music will ultimately be required to have people with PhDs teaching them? Or DMAs?

M: No. I'm saying that as far as accreditation NASM has recently instituted this requirement. That you have to have somebody with a doctorate of music education on your staff to be an accredited music school. I know that because that led to a rush of hiring three new music ed directors in the schools in Spokane where I teach at. Because none of them had somebody with a doctorate. They want to maintain their accreditation so all of a sudden these positions were created and these searches were done. With that comes this idea of moving towards more modern bands, pop ensembles, and songwriting classes. By the way, I have co-taught a songwriting class with my wife. I'm totally on board with that. I think its great. What I'm saying is there's, not a loosening, that's the wrong word. As long as we maintain some academic rigor I think it is fantastic to have different genres. More biodiversity is better. 

J: In classical guitar now, more and more people, as I am sure you're aware, are staying in to get a DMA degree. Just because there are really a lot of classical guitarists with masters degrees. It's hard to distinguish yourself if that is your terminal degree. And I don't know if I see the real health in getting a DMA in classical guitar performance. It seems like an odd idea in one sense. It would be easy to say ``well, if you didn't figure out how to be become a classical guitarist after getting a masters degree maybe you're not made for it. Sticking in school is maybe not the thing for you.'' But I can see that people are staying in school for classical guitar just to have a higher terminal degree when they apply for jobs. And I can also imagine that in the future, hiring classical guitarists won't be based so much on how they play as what degree they have. Because schools will be competing to look good in their catalog with DMA degrees. But I don't know how that relates to finger-style guitar.  Because in our world there aren't a lot of people that have credentials who also have an interest in finger-style guitar. 

M: Well, these things are all cyclical and subject to timing. Which is why I say that—when things are talked to Will about a masters in finger-style guitar right now and you can correct me if I'm wrong but I believe the masters \emph{is} the terminal degree in finger-style guitar.

J: Correct.

M: Do you have a terminal degree in your discipline? Is your discipline something that will attract students. I think the answer to both of those is ``yes'' and ``yes.'' There's employment potential if you want to go the university-route. This is something where we're working on fitting in a discipline that probably predates the human species, which is to say music, into an academic environment. In a lot of things, the degree is, if you're philosophy researcher [unintelligible] that terminal degree makes a lot of sense. We have a constant friction with administration, gently, some of our best players are people that may not even have a masters. They're here because they have a chair position in the symphony. So they are clearly the top trombone player. The reason they didn't continue on with a masters is their students say ``you got a symphony gig. Thank god you don't have to go spend money on a masters and a doctorate.'' It really boils down to skill-based versus credential-based, and that is not say that—I can think of a couple of people who have doctorates in guitar that are smoking players. I'm just in awe of them. And that also be [unintelligible] of doctorates like well you know, not going to performance is their strong suit. And that's okay. Bottom line though is if you're going to be in music, I think being responsible for producing a sound that people want to hear is the only way to be employed. And that could be being a great composer. It could be having a gift for conducting a middle school choir. It could be being a performance guitarist, violinist, it could be a lot of things. But ultimately if you can't produce a sound then, you know, maybe there are other things that would—you know, mechanical engineering would provide a better starting salary. Does that make sense?

(30:23)

J: Yeah. 

M: And one other thing about the finger-style thing. Is there a big difference between finger-style guitar and classical guitar? In some ways, synaptically and biomechanically they're very similar. And where is that music. Is Pierre Bensusan a finger-stylist with what he's doing? And is Andrew York a classical guitarist with what he's doing? It's not a bright line. There's a very ephemeral veil between the two disciplines. And then pop guitar. Is the guy that plays really heavy jazz, you know is into John Scofield, is that guy the pop guitarist or is somebody like Laurence Jubar is when he is arranging Beatles is it? Where do we draw these lines? The bottom line is if you want to teach a university, can you inspire students to produce good sounds and do you have a piece of paper that can back it up? You can probably find a job somewhere.

J: Okay. Any other thoughts?

R: No. 

J: Listen. Its nice talking with you. Thanks for spending some time with us. I hope we'll be able to talk further and maybe run into each other one of these years.

R: Thank you very much.

M: That would be nice. Been a long time since I've been in the Midwest.

(31:31)

J: Do you ever get this way? 

M: We have in the past. It would be fun to come out and do something. As long as we are on the subject, we'd love to come out and do something at the school. There doesn't have to be a bunch of dough involved. It'd be fun to do something where we are covering four hundred years of voice and guitar including original finger-style guitar—finger-style and voice material, classical material. Maybe something can be done there.

J: Do you do classical voice and guitar?

M: Yeah.

(31:57) 

J: Like what kind of repertoire?

M: The usual stuff. We do a lot of stuff written for us. But we'll do stuff like Clive[?], Corderro, Dowland, and Bellini, and all that stuff.

K: Barber.

M: Barber. Samuel Barber. That was my wife, by the way. We also write our own music. Keep in mind. We've been at this a long time. We have been married 24 years and have been singing and playing together longer than that. So our public performances where there are like—for summer festivals—we'll have a Barber tune, a Bellini tune, four bards in DADGAD, some Jetro Tull covers, maybe a Zeppelin tune. There are only two kinds of music: the good kind and the other kind. And we do the good kind. 

(32:39)

J: That might be fun. I put in a proposal to bring in Michael Chapdelaine. Speaking of players that are a little bit tricky to pigeonhole coming up this fall, it might be nice to have you come this way some time. 

M: All you have to do is ask. We'll work it out. We've got a place to stay with Will and all that. By the way, Michael, I mean I hired him for the first Northwest Guitar Festival I hosted and he had a headlining spot. And he was very controversial. Because in the world of classical music he is a controversial guy. And I'm like ``oh, this is so healthy.''

J: When was that?

M: 2007, I think. It was definitely a classical guitar festival. I wanted to have some other options and I thought Mike was a fantastic—he certainly got the credentials. You know his place was to play loud and do pop music.  Some people loved it. It's the typical thing, where in his recital three quarters of the room got up and moved closer and one quarter left. It was that kind of…

J: Yeah. I'm thinking along the same lines. It would be nice to have him come in here just to shake things up a little bit. 

M: Sure. Sure. And plus he is a great player. He's a great guy. I love being around him. So Rachael I want to know what you're plans are. Talk to me about what you want to do.

R: I'll be graduating in tens days or whatever. My family and I are moving out to Boise, Idaho. Which I think is maybe six hours or so south of Spokane.

M: Yep. 

R: I'll be working on an album of original finger-style guitar music, which I have about half-done already. And I'll be basically pursuing a performance career with the aim to either do the performance thing or/also find a position at a university and hopefully make the network of finger-style guitar in academia just a little bit bigger. 

M: I think that's a great idea. Out of curiosity, why Boise?

R: That's a good question. My mother and her partner offered to let my wife, my son, and I live in the first floor of their house. It was basically a financial decision. Looking at Boise, it actually seems that it might actually work to my advantage, as well. I haven't found too much finger-style guitar going on in Boise. I think there is one player out there that I'm aware of. I don't remember his name but my mother has a couple CDs of his. 

M: Yeah. His naming is escaping me too. We've met him and have one of
his CDs. And he's probably ten years older than I am. I'm not sure
with Boise and whether Joseph Baldisari is still there at the state
university but there is bunch of colleges around there. And Boise has
been very friendly. Dan Schwartz has played regularly down there back
in the day when we were in the band and not living in a place with
four walls. We used to go there fairly regularly and play Boise
State. Albertson College in Nampa. There's a bunch of small places out
there. Its become a little bit of an affluent tech community so don't
turn your nose down at doing private teaching. The reality of the universities is that as we move toward the more temp worker model just like everybody else, like law enforcement in the US and postal service, and probably medicine and everything else. You can make as much money teaching privately. No harm in that. Now that said, I'd love to see another finger-style guitar program, specifically finger-style, that would be extraordinarily cool. Because again there aren't that many of them. So I would encourage you to do that. Boise is very cool. There are some good places there. And Moscow is just a little bit north of there. McCall is a little bit north of there. Spokane's not that far away. You want to come and play, you let me know. We can make something happen.

R: Wonderful.

M: And the arts are so tough. Somebody has to support the artist until 40 at which point they support the arts after that. That's another thing to say. Reminds me of my favorite joke. Here is one that I will leave you guys with. ``How do you know when a musician has been staying at your house?'' 

R: How?

M: ``They're still there.'' There's nothing wrong with keeping your resources tight. Keep your expenses—your overhead low and just make a name for yourself. If you play well and you play out you're going to attract students. That's just the way it is. This desire to produce sound, again, it predates the human species. It's not going away. I think that's the missing ingredient. I think that people assume that people will be attracted to the piece of paper. And remember when you were in middle school or when you were first—John you know—when our generation you know—the first time I heard Michael Hedges was like ``what the hell is that and how do I do it?'' And this is before John put out his transcriptions. Those were a god send. You get a little red line and you're hard wired to biomechanical skills to produce that. In a way we didn't have Tarrega and [unintelligible] or whatever. That idea that you hear something and you want to tackle it that's what [unintelligible] students. Guitar is really popular. You'll find a way to contribute a lot to the household just by going out and playing. And it won't take that long. It isn't going to happen the first month. But don't be surprised if you're like ``hey, teaching guitar is kind of cool.'' And then the irony is we don't need the university position. And they're like ``Hey do you want to do some teaching here because you have a studio of 30 students that are actually out playing fingerstyle.'' That's just how it works. Everybody likes getting on a [unintelligible] train including the university. 

(39:04)

J: Well thanks again. Its fun talking. I appreciate your sense of humor and your perspective on all of this. Let's be in touch. And if I can ever help you with anything, with your program, let me know, okay? 

M: Well, keep your masters healthy so that I can send students that way. 

J: Yeah. Do you have any that are looking in that direction now? 

M: There's one guy who says ``yeah, I might want to do some of that finger-style thing.'' And its like ``let's get started now.'' People are drawn to what they are drawn to. I have one student who really likes 19th century music and that's awesome. And other people like 20th—some people like Baroque—some people like whatever. And I don't think that you should spend playing music that you don't love because you can't get through the music that you want to do in one lifetime. There's too much. So if something isn't floating your boat find another program. The fact that you have a place to put somebody like Will, you've seen, he's a hard worker. It's nice that there was somewhere that he could go. Otherwise he'd be teaching privately somewhere or performing or whatever or god-forbid working on you know a lute suite in some masters program some where. You know what I'm saying? So keep your masters degree healthy and I'll see if I can send some students your way.

J: Okay. And if I can help you with anything let me know okay?

M: Well, let's do a gig in the fall. We'll come out there and we'll show you what can be done with crush honor [?] stuff 

J: Okay. Let's consider that. 

M: Rachael, look me up when you get to Boise. Its a cool place. And I'll probably be down there in the summer too. There's a little Suzuki class I want to take. Our god-daughter lives down there so, what the hell, might as well make a trip.

R: Sounds good. Sounds good.

J: Okay. Thanks again. 

M: Okay. Talk to you guys later.

J: Ciao. 

R: Bye.   
% Emacs 25.1.1 (Org mode 8.2.10)
\end{document}

%%% Local Variables:
%%% mode: latex
%%% TeX-master: t
%%% End:

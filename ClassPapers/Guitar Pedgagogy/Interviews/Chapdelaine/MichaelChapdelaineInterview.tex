% Created 2017-05-18 Thu 14:34
\documentclass[11pt]{article}
\usepackage{fontspec}
\setmainfont{Times New Roman MT Std}
\usepackage{graphicx}
\usepackage{longtable}
\usepackage{float}
\usepackage{wrapfig}
\usepackage{rotating}
\usepackage[normalem]{ulem}
\usepackage{amsmath}
\usepackage{textcomp}
\usepackage{marvosym}
\usepackage{wasysym}
\usepackage{amssymb}
\usepackage{hyperref}
\tolerance=1000
\author{Rachael Carlson}
\date{Conducted on April 29, 2017}
\title{Michael Chapdelaine Interview}
\hypersetup{
  pdfkeywords={},
  pdfsubject={},
  pdfcreator={Emacs 25.1.1 (Org mode 8.2.10)}}

\setlength{\parskip}{1em}
\begin{document}

\maketitle
\flushleft

\emph{I have attempted to omit abandoned sentences and filler-words where appropriate in order to ensure that this document is as useful as possible. I have also written words without dialect.}

\emph{M: Michael Chapdelaine; J: John Stropes; R: Rachael Carlson}

J: Hey Michael

M: John, how are you?

J: I'm fine. It's nice to see you.

M: Good to see you.

J: Thanks for taking some time.

M: My pleasure.

R: Hi Michael good to see you again.

M: Good day

J: Why don't you pull up a chair as well?

M: Sorry if I look like a street person but I'm freezing.

J: Well what about, aren't you in New Mexico

M: Well, in my building they turn the heater off at a certain day. which should be fine and all of sudden it's snowing today. --- and now they can't turn it back on so I'm have my feet on the remaining parts of my video studio that weren't stolen, which is the lights, --- video lights

J: Well, It's good, I mean, you're a survivor. Listen, I think you know I think you know the scope of this, Rachael's doing a project in which the goal is to try and understand a little bit more about how guitar programs are constructed around the country round the world and maybe to understand how a program in finger-style guitar in particular could become part of the system of universities or college and I know in talking with you it sounds like you've always included you know kind of an open minded approach to guitar within your program but 

(2:00)

maybe we could just start by asking you exactly what degree tracts are available at the University of New Mexico?

M: We have obviously performance degree which is the one where we take the most liberity if someone wants to play other stuff we get to that and then theres music ed track usually they are so busy with their academics that we pretty much to straight ahead Bb classical until their recitals. A certainly composers we work on composing so that draws from pop music it draws from finger-style traditions and then for whatever else interests them and interests their composition teacher

J: so is that a BFA in composition you're speaking of?

M: we have a BFA but its just general. It would be a BM in Composition.

J: okay, and the education bachelor of music is a bachelor of music ed probably

M: correct

J: and then the performance program is that a BFA program?

M: it is a BM

J: okay

M: Bachelor of Music 

J: okay and I notice that the name of another instructor guitar on the faculty

M: would that be mike anthony

J: yes

M: you know his

J: no

M: he's a badass

J: great

M: he's an old jazzer, like from the wrecking crew days. he did all the sessions in LA in the 60s. but his main thing was always jazz they call guys

J: so what does he

M: he retired a long time ago his wife moved here and started the dance studio

(4:00)

J: so what does he do at the university 

M: good question

(laughter)

M: I never see him

j; does he teach guitar?

M: I assume he teaches jazz guitar because we do the jazz guitar program but they won't come near the classical studio they just You know they have that dichotomy of fear and also of superiority so they stay the hell away from me

J: we know about that here as well. We understand completely. Yeah there's a line. Okay. so to what extent you know we use the term finger-style guitar and its got to be one of the vaguest terms in the world. I mean its almost hilarious. But I think we understand each other when we use that term. To what extent would you say finger-style guitar is a part of your curriculum now.

(5:09)

M: It's been growing a lot in the last years mostly because I just took the reigns off of the curruculum. And I decided that I'm just going to teach them whatever they want to know and so that turned into ``oh I'd really like to do this tune that you recorded on this video.'' ``Okay, fine. It's harder than anything you've ever played but you can play it if you want.'' And so they do. And now their starting to come in a lot with their own arrangements and its quite remarkable. The classical thing works so well for our generation because well that was the only alternative. And it had a discipline. And a curriculum to it that was rigid and rigorous. You know like we could walk in and feel like I was really becoming a college student following some program. You know these days it's a lot harder to interest a young student in rigor because they've grown up in an era where the worst thing that ever happened was the fact that they all got an A and they all got a medal they all got a trophy. So it's kind of hard, its hard to have that same discipline about as we did back in the 70s. So I guess that's the long answer to. We're doing it more each semester. For a while it was just one or tune things. But now it is most everybody is doing something instead of the classical rounds. 

(7:00)

J: You mentioned that a while ago that you took the reins off. What precipitated that?

M: My divorce.

J: So that was actually related?

M: Well you know we have these various phases in life when stuff comes and knocks the hell out of us and then we get to decide what to hang on to and what to let go.

J: yeah, I know about that.

M: You couldn't be this age and not. It's not possible. So anyway, I got beat up pretty good for about three years. And lost everything. And then still get to pay her for the rest of my life. And I just decided that life is too short and its too difficult for me to impose my every set of parameters that I find to be important on all my so that's worth that. It been about 8 years

(8:12)

J: Did you get push back from your institution?

M: No they don't care. As long as I don't ask for money, they don't care what I do. They don't want to know. They care that it gets them their rehearsals and their ensembles. You know that ultimately their students all go to law school after they get out. And I don't know if they even play music is a hobby. Also my students stage of music. And they go on and figure out a way that---guitarists are very industrious---they all go on and figure out a way to make a living, and they do. So, The school doesn't care.

(9:00)

I don't even think they care if I just didn't show up for a while. The students might mention it. And word would get around. And really, and I don't mean to sound cynical, I'm not at all I'm glad that its that way. And we're a very poor school. New Mexico is---as you probably know---we're 49th in almost everything but we're trying harder. And certainly in funding for education I think we are 49th or 50th. The worst.

(9:40)

So yeah, its pretty---I have a lot of freedom.

J: So try to talk about the extent to which that finger-style guitar is currently included. And how its included and when you bring in the repertoire and what kind of repertoire you bring in. 

M: okay, when they start, everybody does kind of the same stuff. They all play they learn in line there's a page you can go look at about. It's on my web page. So you don't have to go looking for it. And it describes a good deal of what we do and it has this list of modes that I want all of my students to play as part of their daily routine. And so they all have to learn that. And then we learn op. 114 by Carulli, which is that set of 24 arpeggio studies. He calls them preludes. And then we ease in a little bit of Leo Brouwer, \emph{Simple Etudes} and then some Sor Studies. Carcassi.

J: Is this on the University website or your own website.

M: I'm trying to find it. So I'll just send you the link.

J: Okay, great.

M: There's a lot of material here. So there it is. I thought there was a place on here where I could send you a note. 

J: Oh, on Skype? You can just---

M: No, I got it.

J: Okay, you can just send it as an email attachment too. 

M: Bam!

J: Okay, let me take a look.

M: So what you'll see is a rather lengthy treatise on how to make tone. And I think it's applicable to both styles, of course. Because good sound 

(12:00)

is a really good thing.

J: Okay.

M: So this thing is called Lesson 1 Focused. I told this is a supply list, that's pretty obvious. And then I wrote another thing about technical things about the left and right hand that I think are universal and pretty helpful. And after that is a set of modes that I made up. Which are just a set of three notes per string I just have a little box for each mode so that you. They are not always three notes but they stay within a four fret area. 

R: Oh I see, yeah. 

M: And so then finally we ease into some Ponce Preludes and then some real repertoire, you know, Bach cello suites and the usual suspects. 

(13:00)

And that's probably two years worth of stuff. Except someone who is pretty gifted already. And then we will start, you know, ``what do you want to play?'' I've got a couple of guys who really surprise me. I've got a guy who brought in a Jason Mraz tune. And did a good job with it. It's not a tune I would  have ever thought to arrange. And you know, I show them how I arrange. We talk about texture. We talk about polyphony. We talk about what to do with melody versus non-melody. And really the teaching in the finger-style is identical to the classical because its still all the same elements.

J: So after kind of two years of a set program you're happy to have people bring in what they are interested or 

(14:00)

make suggestions about repertoire and you help them with it whether you're familiar with it or not you do your best to assist them in following their ambition on this.

M: Right. And very typically they want to play my stuff for which there is a great deal to choose from so that makes it easy. A lot of time they want to play my compositions too which you could call them classical you could call the finger-style it doesn't really matter. It's just polyphonic guitar music. Just like Hedges played. Just like Segovia played. 

J: Do they sometimes work with the music of Michael Hedges?

M: You know, most of them don't know about Michael Hedges.

J: Oh

M: Which is too bad. And I don't really---I can't. It's like delivering them The Beatles or delivering them Chick Corea's \emph{Return to Forever}. You know, its old stuff. Its their dad's stuff. And they just don't care. You know their pretty crazy about Andy McKee and some of the other CandyRat sort of guys. And I rarely will---We certainly will not work with TAB. If they can't find it with real music I won't work on it because, you know, TAB is a great way to look at architecture but it isn't a great way to look at what the design is. You know, I can't point to a 7 and a 3 and say that's a minor 3. 

(15:40)
J: Do you yourself play in unusual open tunings? 

M: No. I only---I never change the top four. And I'll change the bottom two depending on the key. So frequently the sixth-string will end up at a D or a C and the fifth-string might go down to a G. 

J: But it's not a large enough adjustment that you won't have a problem reading it in standard notation. 

M: Correct. Yeah, once I start tuning the rest---I can get by with the---not that we we'd ever use it, but I can get by with a lute tuning too. You know if you turn the G string down to an F\# I could still stay on track because I use to do that so much when I was younger. But once you start doing you know, big boy tunings, Joni Mitchell and Hedges and all that, I just can't do it. I want to know too desperately what I'm playing. 

(16:45)

J: We find now that probably, I don't know, half the repertoire use is in some manner of strange alternate tuning.

M: Right.

J: So all the students---and really tablature is the only recourse.

M: Yeah, no doubt.

J: In that situation. So it becomes much more valuable to students, actually. Or maybe even essential. But we usually write things out in parallel TAB and standard so even if you're reading the tablature you can see the movement of the pitches quite easily. 

M: Right. And you guys all study reading too besides finger-style?

(17:29)

J: Yeah. Yeah. But we don't---we start with foundational elements of finger-style guitar which would include country blues and folk music and every manner of whatnot that has come before that fed into contemporary composing styles for finger-style guitar. So its a different approach.

M: Do you have a curriculum online somewhere that I can see? 

J: No, actually not. But maybe we can talk about it when you come here next semester hopefully.

M: Sweet.

J: If that---presuming that all works out. I think it will, actually. But Rachael has some questions for you. So I'm going to let you take over.

R: Okay. Thanks again, Michael.

M: Hey, my pleasure.

(18:25)

R: So I'm curious about how you first got into teaching guitar.

M: I needed the money.

R: Yeah?

M: I will never forget, I was in college, and my teacher, Bruce Holstman, said ``I've got some students for.'' And I went ``Okay.'' And he says ``What do you teach from?'' And I said ``you know, one of those teaching books.'' He said ``Okay.'' he gave me the number. And I had no idea. I had to go to these little kids house---the dad was a doctor or something. And I just figured it out as I went along. I guess I'm pretty good at it now. 

(19:15)

R: And how long have you been teaching?

M: I guess, 40 years now. 

R: Great. Great. Good work. 

M: Well. Its good news and its bad news. I'm closer to the end. I know a lot more than when I was close to the beginning. 

R: And those first students that you were teaching. Did you teach classical guitar to them?

M: yes

R: Okay.

M: At that time that was really all I had. I mean I played in rock bands before I got into college but I didn't have any kind of vocabulary to teach it. 

(20:03)

J: I assume you were using the Aaron Shearer book or something then.

M: You know, I wasn't a Shearer guy until his final publications with Mel Bay. Which he himself pretty much denounced all of the stuff he had published before that. Which is pretty amazing. No there was no Shearer in our lives at Florida State in the 70s. We used Frederick Noad's stuff. 

J: Oh. 

M: And I told him one time when he was still living I said ``You know I've probably sent your kids to college on the amount of learning the classic guitar''---no, no it's called \emph{Guitar Playing.} Book 1. No. \emph{Solo Guitar Playing.}

J: \emph{Solo Guitar} yeah. 

M: He thanked me and said ``yeah, thanks for doing that.'' Fortunately you're not the only one. But I still use that book when I teach a beginner. 

R: Okay. 

M: Its great. It can't be improved on. 

(21:15)

R: And then, when you started teaching finger-style guitar at the University of New Mexico did you experience any sort of resistance? I mean you sort of hinted at how your institution seems to have a hands-off approach to you. Have you experienced any kind of resistance, either from your colleagues or from the institution? 

M: No. But for a while I kept it in the studio and didn't bring it into the recital but now its fine they play where ever they want. 

R: Okay

J: But you---originally when you got that position they hired you as a classical guitarist, I presume. 

M: Oh God yes. And that's what I was. You know I had no interest in anything else that was it. And that didn't change until after I won Winfield in '98. Even when I went to Winfield, I was a classical guitarist. They just didn't know it. Because the judges can't see you. And they just didn't know. 

(22:28)

J: What did you play at that competition?

M: That was how I pulled this scam off. I played a blues tune that I wrote. And so, that was acceptable under any conditions. I played a waltz by Antonio Lauro. Which sounded a bit, you know, like all those fingerpicker guys were into Chet and into Lenny Breau and into samba rumba stuff and so a Venezuelan waltz by Lauro is---they don't know that's classical music. And so that fooled them. And then I played \emph{El Colibrí} by Julio Sagreras which is just fast and ``Oh my gosh how does he do that?'' And then I played another one of my compositions that's somewhere between rock and roll and Shostakovich. So, yeah, completely bamboozled.

(23:25)

J: Do you think—I know we joke about this but do you really think they didn't spot the Lauro in your program?

M: They didn't know the repertoire. I know who the judges were now because years have gone by and they didn't know that repertoire. 

R: That's a good choice too. From my understanding Lauro is, in a lot of those tunes, he is emulating the Venezuelan harp which is a steel-string harp. From my understanding. So, good on you. 

M: It was a stroke of–the whole thing was like destiny because my life as a classical guitarist was pretty much destroyed by the Segovia incident. And so I was not working nearly as much as I had been before that happened. So I was looking for a way to get more opportunities. And I was reading \emph{Finger-Style} magazine and \emph{Acoustic Guitar} and I kept seeing these like same 5 guys all over the place and they all had this credential which was ``National Fingerpicking Champion.'' ``I guess I better get one of those.'' So I found out it was in this cornfield in Winfield, Kansas. Basically the Woodstock for bluegrass people. I really thought a lot about how I would do it. And what repertoire might work.

(25:00)

And I didn't know any pop music, yet. So that wasn't an option. And I had a handfull of compositions of my own and then a handful of classical things that could sort of be passed-off as something hip. And so, Bam, I got the prize. 

J: How old were you in that Segovia class? Were you in your 20s?

M: I was 28. 29.

J: Oh, you looked quite young. 

M: Yeah, the age gods were good to me until recently. 

R: So how did you get your position at the University of New Mexico? So you mentioned that you were hired as a classical guitarist. 

M: Yeah, in those days there were quite a few professor gigs that would come up. Two or three a year.  

R: Okay

M: I had a part-time gig at Metropolitan State College in Denver that was eventually going to turn into a full-time gig. This New Mexico thing came and so I auditioned and they hired me. 

(26:20)

J: What year was that?

M: That was '85. What I didn't know was they had just fired, two years ago, the most beloved man in New Mexico, who was a Cuban guitarist named Hector Garcia. And he didn't suggest that he was the true Segovia, he made it clear that Segovia was a fraud and that Hector Garcia was the Chosen One to classical guitar in the world.  This state—I'm looking out at it—here it is. Here is my state.

R: Oh wow.

M: This state is half latino. And so this Cuban guy—and he's high Cuban, you know—Cuban's can be so proud and so convincing. So he was really loved more than anybody in the state. When I replaced him—you know, this young-looking 29 year-old white guy from Denver—it was bad. This university of sued by several groups. I was sued. I was harassed until I got tenure six years later. Then they finally just ignored me. 

R: Wow

J: What was the nature of the suit?

M: Well, I would say racial discrimination. That the only true people who could teach classical guitar had Spanish blood. 

J: Why did he get fired?

M: He had some personal problems. He couldn't make it to school anymore. 

J: But he got fired, he didn't quit?

M: Well he had tenure. I don't know exactly what they did but I think that they harassed him to death. And so one day he just stopped going. And he stopped coming long enough that they could make a case to dismiss him. But unfortunately he didn't leave. And so he and all of his compadres just made this young guy's life living hell. I even had my children threatened. I would have people coming to my office regularly threatening me. It was crazy.

(29:00)

R: Now moving to the present. How students do you have now that you might consider to be finger-style students? 

M: Leaving our original statement that everybody who plays polyphony is finger-style I have 15 students. Of them, probably 10 of them are playing some form of non-classical polyphony. And an awful lot of it is again my compositions or my arrangements because they are just readily available. And I like it. 

J: That's material that you've published. So its easily available to them?

M: Easily, yes.

(30:00)

It's become a problem for the pop tunes because I got busted last year by Hal Leonard for my arrangements so those aren't really available anymore. But I think that I can show them to them.

J: My view is that within the context of a private lesson when you and a student are in a private room, you can jot down anything you want and hand it to the student. 

M: That's right. That's still possible. I can't sell them.

J: You know, Hal Leonard is right here in Milwaukee.

M: Maybe you can drop by and put in a good word for me. The problem is years ago when I started making these arrangements, I have well over 100 of them, I asked them and they said ``you're too small, it would cost us too much to administer your publishing than we would make off of it. And so, no you can't have the license.'' 

J: Now, if I remember correctly they have a full-time person just sending take-down notices to YouTube Channels.

M: Oh really?

J: Yeah. 

M: I didn't know that they cared about YouTube. Because they still get paid.  Because they monetize all of the pieces that I didn't write. So I think their okay with YouTube covers. 

J: Maybe so. Do you know how long that='s been possible on YouTube

M: I would imagine since the first time that a Hal Leonard person had a chat with his lawyer. We have something, I forget what we call it, a compulsory license or something that the owner can monetize it and that's what you do in exchange for using copywritten material.

(32:00)

J: Yeah. That's my impression to. However, that wouldn't cover publishing. That would be book publishing would be different entirely.

M: Wow, that's a very contentious area. That's like stealing people's land, they just don't like it. 

J: So I guess that the topic then is what do you think the future will hold for finger-style guitar program's that might crop up in other institutions. Is it likely that they could crop up? Is it always just going to be considered a surreptitious plot that you just have to sneak into a university curriculum? Is classical guitar ever going to open up to welcome more input? Will classical guitar ever get its nose out of the air?

M: You mean out of its ass?

J: What ever metaphor you're looking for. What do you think?

M: I know this is hard to believe but I think the answer is 'no.' I don't think they ever will. I think that because when I get invited to be on a classical guitar festival and its not infrequently that that happens they don't bring me around because I can play classical, they bring me around because I draw an audience. And they have no interest in anything that I do that isn't classical. And even though I'm sort of the obvious link between the two, what most of these guys know—I know most of the people in the classical world, and I guess that I know most in finger-style world—they just don't seem to want to polay together. Its really bizarre. Of course its fear based. The problem for the classical side…

J: I think our signal just froze. Can you hear us?

(skip to 35:28.94)

J: Sorry we got cut off just when you were about to say the most important thing.

M: I was on that really dense V chord with the 9th in it blaring away. I might have forget

J: You know players like Ben Verdery or Andrew York who also have, or even Bill Kanengeiser, people who have, they have marketed themselves as people who do something more something outside the boundaries of classical guitar. And you were saying that you're called upon in that same capacity.

M: The problem they have of course is that an awful lot of the writing in finger-style world is not very sophisticated. They may have very sophisticated technical tools. Combination of percussion and pull-offs and slurs and things they can do is really astounding and they have a fantastic groove and sense of rhythm that coming out of finger-style's heads is wonderful. But most of them can't write. I mean, its your job at Wisconsin to help carry on—it did start with a very good composer, this whole movement with Michael hedges was a great composer. And a lot of times I think there's a mistake that one he did was, he was a great guitar player, and he truly was, but the reason he was so great writer. And if the finger-style world doesn't get its writing a little bit better, of course most of them don't go to college. So I think its [unintelligible] who are going to help them to notice that their not writing, Because most of them are metal guys and basically they are writing metal tunes. You know, there's verse A, Verse B which is Verse A with a little more stuff. there's not much development of material and its not because their not talented but their just not trained.

(38:19)

J: Well, that's interesting that there's some, its circular in a way. If students are not in the academy, if they aren't learning to become better writers\ldots{}Do you think that its healthier for finger-style to stay out of the academy. Because if its out of the academy they don't have to do, they just get to create their whole deal and they can run with it as fast as they can. Or if finger-style guitar is in the academy will it become so well organized that it will no longer be interesting. That would be another aspect of it.

M: Yeah. I think we're seeing saturation certainly of finger-style. And I think we're seeing also the ceiling of composition. And learn Haydn and Beethoven. The problem with everything, and of course the reason that classical music is dead, is because we always have to think better and we have to get more clever and we think of ourselves as artists the same way that physicists, like scientists do, which is that we always have to go find something new. And we saw that happen in painting until abstract expressionism everybody just thought its ugly so they went through and just started going back. And music when we finally get abstract expression, which is basically Schoenberg, even though its highly serialized it all sounded like ass and just because its original it had some excuse to exist in the academic world. And so I guess the thing I'm trying to say is if we don't grow we just disappear, I think. So I think the only hope for finger-style is what you're doing, otherwise it's over. I just saw the monkey guy\ldots{}guitar monkey\ldots{}

R: FretMonkey?

(42:00)

M: Fretmonkey. So you know that guy was so excited about it that he was going to come out and take on CandyRat. And he had a core of five or seven guys and they're all just incredibly skilled. And the world didn't say ``thank you''. It just said ``Huh, here's this again.'' 

J: Did you enjoy some of those players? I'm not familiar with their work.

M: I don't hear a lot of melody. And if I don't hear a lot of melody and if I don't hear a lot of harmony I don't like any music. Even with Lady Gaga if she doesn't find her way around into a melody I start to get kind of eyeing for the exits. And so I think most of them FretMonkey guys who are incredibly good at reproducing the things that are easy to recognize in Andy McKee or Hedges but not necessarily the content that it carried. And so its just of a logging [?] of derivative stuff.   

(43:00)

so yeah, those guys need to go to college.

J: So do you have any recommendations for someone, like Rachael, who might be interested in proposing to, cold calling some university about establishing a finger-style guitar program. Any advice?

M: Well it sounds like you have a real musical background. You play counterpoint and study harmony. If somebody mentioned Late Beethoven String quartets it would mean something to you.

R: Mmmhmmm.

M: I think that you have a chance of at least getting an audience with those kind of folks. Obviously its probably why you're doing what you're doing. 

(44:00)

With going to college and studying guitar. Yes, someone's got to do it. I mean it can't just go away because Hedges died. I mean, he's been dead too long. But it hasn't gone anywhere since he died. If you showed me the best composition of any finger-style guitarist currently working, I would say, ``oh well that's not as good as ``Aerial Boundaries'''' or ``that sounds a lot like ``Aerial Boundaries.'''' I heard one by a really famous guy lately. And fortunately he wrote on there that it was a tribute to Hedges. It was like ``wow, this is Hedges'' well it wasn't it was almost Hedges. 

(45:00)

So you go to a Thai restaurant and they have pasta that isn't always Thai and you're thinking ``wow this is almost Italian.'' Am I right? Has anybody done anything better than Hedges?

J: Well, a couple of people. You know I was thinking too, I don't know if I have your mailing address. If you could send me your mailing address, I'd be happy to send you something that you might find interesting.

M: cool

J: and then I don't know if you will find any of this interesting but we have tried to stay up and we are obviously looking for the best material possible but that's come out in the 20 years since Michael passed away. I can't believe that its been that long. I don't believe it. I refuse to believe.

M: Yeah, its pretty insane. 

R: Sort of thinking more personally about myself. Do you think that its necessary to have a performance and recording career before a University or college will take somebody like me seriously as a candidate for an instructor position?  

(46:35)

M: As far as I can tell, the way it works now is they're looking for a Doctorate and their looking for competition victories. 

R: Okay.

M: And I don't think anybody can get hired anymore without those.

J: Well you're talking about a DMA in classical guitar for example. Or just in miscellaneous PhDs?

M: I think to get a gig teaching guitar at a university you need a DMA in guitar performance.

J: and by that you mean classical guitar performance?

M: Until we find a way to make it broad again. I mean that's a pretty daunting taSK. 

J: Well I think there, the world wound up with way too many with masters degrees in classical guitar. And that was no longer a distinguishing credential.

M: right. 

J: of course in guitar generally the situation is a little bit different. there are hardly any DMA guitar programs in Jazz guitar specifically although there would be degrees in Jazz Studies. More likely. General jazz programs

M: right, I think USC has one, Indiana has one,

J: I think northwestern has a program in Jazz studies or I'm not sure what they call it exactly. Around the world when you travel around the US are there any spots that you think are interested in a broader definition of guitar that you play with your fingers?

(48:55)

M: I see them going the other way. It just seems, its a pedagogy that is so self serving that it creates little Mussolini's, the classical guitar pedagogy. The only thing that will make them change their mind would be the need for money. Because once you've been indoctrinated—and you know, I was very much in the Borg myself for a long time—once you've been in, you're in. It's like being a Moony. You remember the Moony's.

(50:00)

J: Rev. Moon.

M: A religious group back in the 70s. You have everything you need. There's warmth and there's nurturing and theres mutual respect, all these things and they all play at each others festivals. Nobody cares about it. You get really sad.

J: Well that is an interesting comparison.

M: And I love classical music. And it's just over. The only people we see who are really kind of getting ahead are the one's who are creating their own music. We see, this guy who is going to be on the Internation Guitar Night named Marek Pasieczny. He writes his own music and he plays the hell out of his guitar. He has like 45 doctorates. I don't know how he can have so many degrees and not already be in a wheelchair. But he is amazing and he is getting tons of work and its because when he shows up he has something to say that sticks. So I know that classical dies. I know that I will be playing the same fifteen pieces. Its not going to work much longer. So I think, What I am saying is that there is a huge hole for finger-style people who have some level of sophistication to go in and grab work. 

(52:05)

I don't know how its going to ever fit in academic though. 

J: Why do you say that?

M: I think you guys froze. You froze a while ago.

J: Can you hear me?

(skip to 54:05)

J: Can you hear me? I know our video and audio of you is out of sync\ldots{}

R: Are you aware of other universities in the cournty or in the world who offer or include a finger-style curriculum?

M: No. I think we're it. 

J: Okay, well power to the people. Michael thanks very much I really appreciate you taking some time. You've been very generous and it nice to hear to talk. I'm charmed by your philosoophy.

M: I don't want to be depressed. Sometimes when I, I'm so invested in classical music and I just see it, unless everybody starts writing, that's what classical music was in the past, it didn't have all of these trained robots who only played the repertoire that was approved by the grand potentate. People loved music and they learn how to play it. That is the beautiful thing about finger-style, pretty much every one is original in their intention. And if we can figure out a way to convince them that it would be good for them to study Beethoven instead of just the finger-style guys that they love there is no question that a kindred spirit is in all of this. There is not one that is necessary more weak than another, you know. If you give them the training they're going to turn out to be great. And I know you can appreciate this because you have been doing this as long as I have because you have been trying to take this out of the classical box. You're doing the right thing. 

J: Thank you. I appreciate your thoughts greatly. Will you send me your mailing address?

M: I will.

J: And I would love to send you a package of stuff that you might find interesting. And we can continue this conversation
% Emacs 25.1.1 (Org mode 8.2.10)
\end{document}
%%% Local Variables:
%%% mode: latex
%%% TeX-master: t
%%% End:

\documentclass[12pt]{article}
\usepackage{fontspec}
\setmainfont{FreeSerif}
\usepackage[margin=1in]{geometry}
\begin{document}
\noindent Rachael Carlson\\
\noindent Guitar Pedagogy\\
\noindent Finger-Style Guitar Method Comparison\\
\noindent \today\\
\strut\\
\raggedright
\noindent Prompt: These two methods each contain general information on the guitar and its history, explanations of technique and notation, and repertoire including original compositions by the author and public domain compositions that have been arranged by the author.\\
\strut\\
\noindent What do you consider to be the best parts of each? The worst?\\
\noindent In the future, would you use these books in your teaching?\\
\strut\\
\noindent I feel that the best part of the de Grassi Method are the sections which deal with the intricate technical details of performance practice in his tunes. I feel that I will use these details when discussing his compositions with students in the future. In particular, these notes will be beneficial for the student if he or she is studying one of Alex de Grassi's compositions. The worst part of de Grassi's method, to me, would have to be the material at the beginning of the book. The explanatory pictures are of little to no use. The text is cryptic. The typesetting of the music can be potentially confusing to a beginning player.\\
\strut\\
In \emph{The Guitar Book} by Pierre Bensusan, in the Glossary section, it is stated that
\begin{quote}
changes in the left hand on the fret board are reminiscent of a cat slinking across a cluttered table without overturning a single object. At every step, it will choose the best spot to place its paws. [¶] Thus, it is important that the player remain curious and not limit himself to what the fingers are familiar with or feel like doing at a given moment. Whatever the music, any new approach to fingerings is beneficial. (20)
\end{quote}
I would say that this is the best part of the Bensusan method. Some of the language that Bensusan uses in the Glossary section is metaphorical in a way that speaks to me as a musician. Similar to the manner in which Chopin inspired his students with descriptive language, I believe that this section aims to do something similar. The music, while not exactly a representation of the recordings that are referenced at the top of each tune, are well executed and proceed progressively. I believe that a beginning finger-style guitar student would be able to work with some of these tunes and develop his or her interpretation of Bensusan's music with increasingly difficult repertoire.\\
\strut\\
I personally have a penchant for Bensusan's music. I anticipate that I will continually refer to his book in the future with students as a reference, a source of repertoire, and a source of enjoyment. I am not too sure about the de Grassi method. I may utilize some of the examples which explore techniques used in de Grassi's compositions. I don't see myself using for any other purpose.
\end{document}
%%% Local Variables:
%%% mode: latex
%%% TeX-master: t
%%% End:

\documentclass[12pt]{article}

\usepackage[margin=1in]{geometry}
\usepackage{fontspec}
\setmainfont{Adobe Garamond Pro}
\author{Rachael Carlson}
\title{Guitar Pedagogy Final Exam}
\date{\today}

\begin{document}
\maketitle

\section{Matanya Ophee}
\label{sec:question-1}

A major point that Ophee makes in the introduction of the his
examination of Spanish language guitar methods is that the methods
take the perspective that the student does not have a teacher. The
methods are designed to take the student from relatively
beginner-level status to some level of mastery without a
teacher. Another important thread that Ophee identifies, though not as
directly as he could have, is that each of these methods want the
reader to take the guitar seriously. In the method by Ferandiere,
\emph{Arte de Tocar La Guitarra Española} (1799), Ophee states that
Ferandiere that the guitar ``must not be played only for the
accompaniment of fandangos and boleras, but regarded as a serious
instrument for the performance of chamber music.'' This statement is
taken as assumed in each of the other methods. The `evolution' of
these methods seems to be centered around complexity. The earliest
methods examine the principles of playing the instrument; the later
methods attempt to take a more extensive, progressive approach to the
instruction. The student starts with relatively benign etudes and
exercises and progresses toward more difficult repertoire eventually
working on mastering concert pieces. In speaking about the `evolution'
of these methods, Ophee states that
\begin{quote}
  there is a qualitative difference: the newer pedagogies found new
  ways to talk about the guitar and its technique. These new ways of
  communication were necessary in order to adapt to a faster, changing
  world. But the actual, physical business of teaching a student to
  move a finger across a stretched string or placing another on the
  fingerboard, has changed little.
\end{quote}
He speaks here, mainly about the changes in how one talks about the
instrument. This is a necessary component of a method.

\section{Michael Chapdelaine}
\label{sec:michael-chapdelaine}

It is interesting to think about pedagogy as a mountain or as a
turn. I am not entirely sure that it is an accurate or useful
metaphor. When approaching this question, I feel that I must use the
same rubric for the examination of finger-style guitar pedagogy as
Chapdelaine does for classical guitar pedagogy. I think that
Chapdelaine mentioned the large amount of importance that is placed on
technique in the classical guitar pedagogy. I take this to mean tone
production. If I were to use this approach, I would have to say that
finger-style guitar is not yet at its apex. We have not had any real
in-depth examinations of tone production. We have not had an Aaron
Shearer figure. The technique of playing a steel-string acoustic
guitar has not been codified as it appears to have been in the
nylon-string world. I think that we are at the bottom of the mountain
in finger-style guitar pedagogy. If we disallow classical guitar
pedagogy as contributing to finger-style guitar pedagogy, we are in
the beginning phases of its development. We have not had 200 years of
pedagogical study from which we can draw. If we all classical guitar
pedagogy to contribute, \emph{we are} the reorientation that Chapdelaine
mentions.

It would have to take an extreme turn in order for the mountain
metaphor to be considered appropriate for finger-style guitar. We seem
to take an entirely different approach to the performance of the
instrument. Especially here at UWM, technique is not necessary if some
one is able to produce acceptable results.

\section{Eigeldinger and Nyberg}
\label{sec:eigeld-nyberg-kell}

One major similarity that I see between these two readings is the
willingness to change, or be malleable for a given student. Set
curricula does a student no good if that student is not getting it. It
seems, from Eigeldinger's introductory remarks on his research into
Chopin as instructor, that Chopin cared deeply for ensuring that each
of his students understood a given item, such as melodic phrasing. In
order to ensure that his student were understanding what he was
teaching, he would change his language when it seemed that a student
wasn't comprehending the previously used language. It seems to me that
Nyberg is recommending a similar approach to the instruction of
students.

\section{The Canonization of Finger-Style Guitar Pedagogy}
\label{sec:canon-fing-style}

This question is interesting especially following the previous
question. It almost begs the question, ``is it \emph{ever} beneficial
to canonize anything?'' I think that if we canonized finger-style
guitar pedagogy right now, we would immediately recognize our error
--- we would go the way of the classical guitar and no longer be
relevant. I think that it might be beneficial to take a long,
difficult look at how we instruct finger-style guitar as a means to
perhaps establish some sort of goal orientation. This might take the
form of ensuring that students are able to perform a wide variety of
finger-style guitar repertoire. I think that we might feel like failed
teachers if our students are only able to play one sub-style of
finger-style guitar, like left-hand over technique. I think that it
would be beneficial to examine each student individually and assess
his or her weaknesses and strengths and find a way to reduce the
weaknesses and build upon the strengths. Is this canonization? I don't
think so. Canonization of the pedagogy might look more like requiring
that in order for a student to reach the next level of repertoire she
or he would need to perform a particular composition at a certain
level. There would be ten or twenty or a quarter-million levels with a
particular piece which must be performed in each level. I do not think
that this would be advantageous for the student, the teacher, or for
the genre.

An interesting side note, composition is an integral component of
finger-style guitar. How would the canonization of finger-style guitar
pedagogy interact with this element? Would each teacher of
finger-style guitar also need to be able to versed in composition
pedagogy? Is composition pedagogy canonized?

\section{The Most Important Characteristics of a Successful Lesson}
\label{sec:most-import-char}

There are several important characteristics of a successful lesson
which can be identified through an analysis of the supervised teaching
sessions we had this semester. A major characteristic is the teachers
effective use of language. The instructor must be away when he or she
is communicating about specific details. For instance, when talking
about the fingers of the hands, the instructor should use the
identifying names that are used for each finger. It may be beneficial
to tell the student use your first finger of your left hand, your
first finger is your index finger of the your left hand. The effective
use of language is even more important when instructing a student
who's first language is not the primary language used in the private
lesson. The instructor should not use language in a haphazard manner
when translation needs to occur.

Another important characteristic that I can identify is that the
instructor should not expect to get to everything that he or she
prepared for the lesson. The instructor needs to be able to devote
more time or energy to a certain passage or musical event if it is
obvious that the student requires this time or energy. It would be a
disservice to the student to gloss over such a thing simply because
the teacher wants to be able to get to everything.

\section{Finger-Style Guitar Existentialism}
\label{sec:finger-style-guitar}

\emph{Finger-style guitar doesn't exist; it just derives from classical guitar.}\\

\noindent I do not believe that this statement is
defensible. Finger-style guitar does not share the same history as
classical guitar. Would a classical guitarist say that Blind Blake
derives from classical guitar? My guess would be that that guitarist
wouldn't want to be associated with Blind Blake. I think that
classical guitar finds itself as associated with written traditions of
music. Finger-style guitar seems to have a stronger link to oral
traditions of music than it does written traditions. This can be
easily observed by composers of finger-style guitar and their
propensity to not write down there music in a prescriptive manner.

I am actually surprised that someone would say that finger-style is
derived from classical guitar. It seems more to me that this person is
either trying to get finger-style guitar students or talk about how
great classical guitar is. I think that a similar statement could be
made of classical guitar if you examine its history in Spain. It seems
to me that it might be derived from an oral tradition just as much as
it is derived from a written tradition. So, classical guitar doesn't
exist; it is just protected from the rain by the umbrella of
finger-style guitar.

\end{document}
%%% Local Variables:
%%% mode: latex
%%% TeX-master: t
%%% End:

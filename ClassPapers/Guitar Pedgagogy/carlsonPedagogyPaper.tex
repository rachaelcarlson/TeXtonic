\documentclass[12pt]{article}        

\usepackage{fontspec}
\setmainfont{FreeSerif}

\usepackage[margin=1in]{geometry}
\title{Onward: \\ Bringing Finger-Style Guitar to Other Universities}
\author{Rachael Carlson}
\date{\today}

\begin{document}

\maketitle        

\section{Introduction}
\label{sec:introduction}
The world of finger-style is faced with a beautiful problem: there is
currently only one university program in the world which offers a
bachelor's level degree in finger-style guitar. It is beautiful
because there is a program which has the foresight to offer a
program. It is a problem because the fate of finger-style guitar in
the university is intimately intertwined with the success of this
program and its graduates. From an extreme viewpoint, this program
could be viewed as either the catalyst for the flourishing of
finger-style in the university or the death knell for finger-style
guitar. The intention of this paper is to establish a foundation upon
which a proposal for a similar finger-style guitar program at a
university can be built. This proposal can take many forms. I will be
examining numerous possibilities for how a proposal could be constructed.

This topic is massive. This paper is a personal journey and it can be
reflected in the language used in this paper. I am not going to be
able to address each detail of a proposal to a university. While
conducting the research for this paper, my aims shift, my tactics
change. For instance, after speaking with Michael Chapdelaine, my
vision of the future has changed. I plan on redoubling my efforts as a
performer/composer as my avenue by which I may enter academia as a
finger-style guitar instructor. Such changes as this are not to be
viewed as hindrances to the completion of a paper. They are deeply
related to the purpose of this paper.

\section{Tactics}
\label{sec:tactics}
This section will examine the different manners in which it seems, to
me at least, that I might be able to convince a university to offer or
reveal the importance of offering an undergraduate degree in
finger-style guitar performance. The first is to situate the
instrument and its manner of performance within its contexts as a way
to show the historical significance of the style. Second, I will
attempt to look at how the style is existing in the present and how it
might exist in the future as a way to anticipate its future
successes. Finally, and perhaps most importantly, I will discuss
different ways in which I can development as a performer, composer,
and scholar as a means to establish my own significance in relation to
the style. I deem these to be tactics which I should have in my back
pocket, if you will, in case I am ever in a situation in which I might
need to demonstrate the significance of the style or of myself.
\subsection{Situation}
\label{sec:situation}
This tactic takes two forms: first, the history of the guitar; second,
the history of finger-style guitar as a genre of music. I will not
spend too much space in this paper going into a detailed history of
either of these subjects.

The purpose of situating the guitar within its contexts serves
multiple purposes. First, it reinforces my own understanding of and
abilities to discuss the historical components of finger-style guitar
as a manner of playing the instrument and as a genre. Second, this is
an area in which the scholarship is not as robust as it could be: the
history of finger-style guitar is yet to be written. Third, deep
knowledge of these subjects may aid in the pursuit of revealing the
significance of finger-style to individuals who do not know anything
about the guitar or finger-style in particular.

\subsubsection{The History of the Guitar}
\label{sec:history-guitar}
This section of the situation tactic will go into detail on the
development of the steel-string six-string guitar along with an
examination of its predecessors, the seemingly mythic \emph{guitarra
  latina}, the renaissance guitar, the vihuela, the baroque guitar,
and the early six-string guitar. This section would be similar in
focus and breadth to the Finger-Style Guitar: History \& Performance
class offered at the University of Wisconsin-Milwaukee by John
Stropes.
\subsubsection{The History of Finger-Style Guitar}
\label{sec:history-finger-style}
This is one of most exciting areas of interest to me as a scholar. A
history of finger-style guitar has not been written as of the writing
of this paper. I have been formulating theories on how to discuss the
history of finger-style guitar. The most convincing way, to me, to
discuss the history borrows from studies on feminism.\footnote{It
  appears that Martha Lear of \emph{The New York Times} was the
  individual who coined the term `first-wave feminism.'} In the
historiographies of feminism, authors frequently use the term `wave'
to describe groups of contemporaneous feminists (i.e. 1st wave
feminists, 2nd wave, etc.). In the case of finger-style guitarists, I
believe that the history would begin with proto-finger-style
guitarists who could be considered the first wave. These
proto-finger-style guitarists would be individuals such as Arthur
`Blind' Blake, Huddie `Leadbelly' Ledbetter, Charlie Patton, Lonnie
Johnson, Robert Johnson, Maybelle Carter, and others who were playing
the guitar with their fingers and not a plectrum during this era. If I
were to produce a paper on this subject, I would most likely start by
limiting the scope to the history of finger-style guitar as a genre in
the United States. Even limited the scope to finger-style guitar as a
genre might suffice.\footnote{The plectrum-style compositions of
  Michael Hedges propose a curious case in this respect. Perhaps not
  dissimilar to proto-finger-style guitar compositions such as
  ``Guitar Chimes'' by Blind Blake, plectrum-style compositions by
  notable finger-style guitarists could be considered finger-style in
  an honorary sense.} It is important to understand one's historical
contexts. It is from these perspectives that one might be able to
anticipate future developments of the genre.

\subsection{Projection}
\label{sec:projection}
I see this tactic as having two subsections, the near-present and
possible futures. The near-present would examine the changing
landscape of the business of music. More specifically, I would attempt
to find some numbers, real data, on the presence of finger-style
guitar on YouTube and other internet sites as a way to establish the
ubiquity of finger-style guitar. This information would help project
the status of finger-style in the future.  The goal of this research
would be reinforce my perspective toward the future in a way that I
can energize those around me to the unique possibilities of offering a
finger-style guitar performance degree.
\subsection{Personal Growth}
\label{sec:personal-growth}
After the interview with Michael Chapdelaine in which he said that he
thought that I would need a D.M.A. in order to be hired as a
tenure-track faculty member in guitar performance, the personal growth
tactic become much more important to me. I realized that Chapdelaine
might have been referring to classical guitar instruction in relation
to his recommendation of a D.M.A. This leads me to believe that one
way into academia as a finger-style guitar instructor could be through
establishing myself as a performer and composer within the genre. This
could perhaps fulfill an experiential requirement for employment.
\subsubsection{Performer}
\label{sec:performer}

\subsubsection{Composer}
\label{sec:composer}

\subsubsection{Scholar}
\label{sec:scholar}


\section{Intitutions}
\label{sec:intitutions}

\section{Interviews}
\label{sec:interviews}

\section{The Speech}
\label{sec:speech}

\section{Conclusions}
\label{sec:conclusions}


\end{document}
%%% Local Variables:
%%% mode: latex
%%% TeX-master: t
%%% End:

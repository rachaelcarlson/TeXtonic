\documentclass[12pt]{article}

\usepackage{fontspec}
\setmainfont[Ligatures=TeX,Mapping=tex-text,Numbers=OldStyle]{Garamond Premier Pro}
\setmonofont{Garamond Premier Pro}
\usepackage[]{biblatex-chicago}
\addbibresource{references.bib}

\begin{document}
\noindent Rachael Carlson\\
\noindent \today\\
\noindent Book Report: \emph{Damiano} by R.A. MacAvoy\autocite{damiano}\\
\noindent Master Class in Finger-Style Guitar\\

This book is centered around the character Damiano Delstrego, a young
wizard and musician living during the period of the Italian
Renaissance. Damiano is a student of the archangel Raphael who teaches
him lute. Damiano’s hometown of Partestrada is visited by the Roman
General Pardo who sends his troops to find the inhabitants of
Partestrada. This leads Damiano to strike a bargain with the Devil in
order to try to ensure that Partestrada will exist in peace. Damiano
then goes back on this bargain by asserting his love of God.

This book is a struggle. I wanted so badly for it to be a good
book. At the end though, I came to the conclusion that it is just not
what I wanted it to be. It is poorly written: there is little to no
development of setting or characters, moral philosophy is hinted at
but never developed, and the plot is weak. I chose this book for the
book report in the hopes of finding some great examples of clear,
lyrical writing about plucked-string instrument performance. Sadly,
all of it is contained in the first chapter. I hesitate to go further
into the trilogy as it does not reveal a potential sense of
satisfaction. On page 6 and 7 of Damiano is the only really
interesting writing. This is when Raphael takes the lute from Damiano
to demonstrate that simplicity can be an interesting component of a
performance.

Roberta Ann MacAvoy was born in Cleveland, Ohio in 1949. Her first
book, \emph{Tea with the Black Dragon} is quite highly regarded. It
was nominated for the Nebula Award in 1983\autocite{nebula} and the
Hugo Award in 1984.\autocite{hugo} It won the Locus Award for best
first novel in 1984\autocite{locus} and helped MacAvoy win the John
W. Campbell Award for Best New Writer in 1983. Interesting to us as
finger-style guitarists, R.A. MacAvoy suffered writer's dystonia. In
an interview with \emph{Lightspeed} magazine, MacAvoy that it is a
\begin{quote}
  rare neuromuscular disease characterized by paralysis and
  pain...[with] almost no research as to the cause of it. My own guess
  is that I came off too many bucking horses in the mountains and
  landed on the back of my hea, or that I took too many spectacular
  falls on the kung fu mat, with similar results.\autocite{lightspeed}
\end{quote}
Similar to Billy McLaughlin's much-documented battles with Focal
Dystonia, R.A. MacAvoy used Botox injections as a way to paralyze the
muscles that cause the permanent `charley-horse.' In 2011,
R.A. MacAvoy released her first novel since the diagnosis of dystonia,
\emph{Death and Resurrection}.

\printbibliography
\end{document}

%%% Local Variables:
%%% mode: latex
%%% TeX-master: t
%%% End:

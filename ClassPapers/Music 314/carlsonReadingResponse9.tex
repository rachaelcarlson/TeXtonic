\documentclass[12pt]{article}

\usepackage[margin=1in]{geometry}
\usepackage{fontspec}
\setmainfont[Ligatures=TeX,Mapping=tex-text,Numbers=OldStyle]{Adobe Garamond Pro}
\setsansfont[Numbers=OldStyle]{Roboto}
\setmonofont{Adobe Garamond Pro}
\usepackage{hyperref}
\usepackage[]{biblatex-chicago}
\addbibresource{bibliography.bib}

\begin{document}
\noindent Rachael Carlson\\
Music 314\\
Reading Response 9\\
\today\\

\noindent \fullcite{pinch}.\\

This article certainly reads like it was written by fans of music instead of scholars of music. It also reads as if the authors were not aware of gender studies or queer studies. While that is the case, I do not believe that the authors were inappropriate in their treatment of Wendy Carlos as a trans performer and producer of music. However, I do think that they were inappropriate with how they treated women in electro-acoustic music. This is primarily in relationship to Rachel Elkind and Wendy Carlos. There seemed to be hint that the relationship between women and S-OB was due to the nature that Rock was a boys game and electronic music was a way for women to enter into the rock scene. I think that this issue could have been more developed. It would be curious if this subject is discussed in other parts of the book. I would be interested in reading more about the status of women in rock music in the late 1960s and early 1970s. The first major woman rock figure that I can think of was in 1975 with Patti Smith.\\

\noindent \fullcite{nickleson}.\\

How cool to read the name Glenn Branca from a scholar.

I don't think that the author of this article is very connected to the title of the article. While it is apparent that this discussion is on the community surrounding minimalism, there is not much mention of that beyond the introduction. I think that it would be better to title article with something related to the process of minimalism and perhaps how it relates to a subversion of traditional processes. There is not much that I find interesting in this article. What interests me is what this article seems to take for granted. My guess is that the author assumed that I have read the other interesting discussions surrounding minimalism and process. I think that I would want to further research this area. 

\end{document}
%%% Local Variables:
%%% mode: latex
%%% TeX-master: t
%%% End:

\documentclass[12pt]{article}

\usepackage[margin=1in]{geometry}
\usepackage{fontspec}
\setmainfont[Ligatures=TeX,Mapping=tex-text,Numbers=OldStyle]{Adobe Garamond Pro}
\setsansfont[Numbers=OldStyle]{Roboto}
\setmonofont{Adobe Garamond Pro}
\usepackage{hyperref}
\usepackage[]{biblatex-chicago}
\addbibresource{bibliography.bib}

\begin{document}
\noindent Rachael Carlson \hfill \today\\
Music 314 \hfill Due Date: April 17, 2017\\
Reading Response 10\\

\noindent \fullcite{auner}.\\

\noindent This article seems to be a survey of post-Integral Serialism
music. It seems to be attempting to push the term ``texture music'' to
describe this kind of music. I wish that I had a better understanding of the
differences between modern music and postmodern music because it seems to me
that this is what the author is really talking about. It seems that the push
away from a traditional melodic/harmonic/rhythmic sensibility, this includes
12-tone as it is really the furthest one can take these traditions, is
indicative of the postmodern turn. Auner takes the stance that the postmodern
turn is displayed by what he calls ``texture music.'' 

I wish that Auner would have talked more about Gubaydulina and some of the
other proto-minimalists. It seems to me that he mentioned Gubaydulina as a
token female composer. Also surprising is the lack of discussion of Nadia
Boulanger who really seems to have had the most profound impact on twentieth
and twenty-first century musics.\\

\noindent \fullcite{bernard}.\\

It is interesting to read this article after reading the above article. It
seems that the (post)minimalist turn is still a turn away from the traditions
of 12-tone and integral serialism. It is a turn away from melodic
sensibilities toward a stronger sense of harmony. It seems to be about
vehicles. In previous traditions of western art music, melody seems to me to
have been the primary vehicle driving a composition. Rhythm and harmony take a
backseat to melody. In minimalist and postminimalist music, harmony, either
tonal, post- or quasi-tonal, drives the vehicle while rhythm is in the trunk
and melody is standing at the street corner. 

Here is another article in which women are left out. I am quite surprised to
see some composer I have never heard of (Michael Torke) while a composer whom
I believe fits quite neatly into the conversation on postminimalism --- Joan
Tower. I can't help but wonder why some people are considered in a canon and
others are not. I would be curious to find more composers who fit into a
minimalist or postminimalist framework who are not white men.  
\end{document}
%%% Local Variables:
%%% mode: latex
%%% TeX-master: t
%%% End:

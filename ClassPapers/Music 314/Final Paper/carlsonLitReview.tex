\documentclass[12pt]{article}

% \usepackage[margin=1in]{geometry}
\usepackage{fontspec}
\setmainfont[Ligatures=TeX,Mapping=tex-text,Numbers=OldStyle]{Adobe Garamond Pro}
\setsansfont[Numbers=OldStyle]{Roboto}
\setmonofont{Adobe Garamond Pro}
\usepackage{hyperref}
\usepackage[]{biblatex-chicago}
\addbibresource{references.bib}

\usepackage{multicol}

\author{Rachael Carlson} 
\title{Music 314 Literature Review: Early Music Performance Practice Discussions From 1980 to 2000} 
\date{\today}
\begin{document}
\begin{titlepage}
  \maketitle
  \pagenumbering{gobble}
\end{titlepage}
\pagenumbering{arabic}

\section{Introduction}
\label{sec:introduction}
Of the easily accessible articles on the subject, there are two
primary discussions that can be analyzed: first, the debate between
whether or not early 15th-century chansons were written with
instrumental accompaniment; second, two major definitions of the word
\emph{authenticity} as it relates to early music performance. I would
imagine that once one digs deeper into the literature there are
debates about the specific performance practices of different genres,
regions, and instruments. As such, this paper will primarily focus on
the definitions of authenticity forwarded by individuals such as
Richard Taruskin and John Butt in the 1980s through the 2000s only.
This is a massive area of research. I have attempted here to highlight
these most pertinent sources available to me.

This literature review ended up relying heavily on Michael Troy
Murphy's dissertation from 2008 on the recordings of J.S. Bach's
\emph{Passio Secundum Johannem}.\autocite{murphy2008} Murphy traces
the interest the performance of historical music from Mozart arranging
Bach and Handel, to Dannreuther Dolmetsch in the early twentieth
century, to the founding in 1933 of the Schola Cantorum Basiliensis in
Switzerland, to contemporary scholarship on performing historical
music.\autocite[9--16]{murphy2008}
% Murphy notes that Dorottya Fabian,
% in ``The Meaning of Authenticity and the Early Music Movement,'' comes
% to the ``conclusion that musicians and scholars in continental Europe
% came to regard `authenticity' as a futile utopian attempt fifteen
% years before English speaking countries.''\autocite[23]{murphy2008}


% \section{Heresy}
% \label{sec:heresy}
% A major discussion occurred in the 1990s surrounding the performance of 15th
% Century chansons. The major individuals discussing this were
% Slavin\autocite{slavin1991} and Urquhart\autocite{urquhart2011}.

\section{Issues of Authenticity: Taruskin vs. Butt}
\label{sec:auth-vs.-inform}

In 1982, Taruskin wrote that we ``tend to assume that if we can
recreate all the external conditions that obtained in the original
performance of a piece we will thus recreate the composer's inner
experience of the piece and thus allow him to speak for
himself.''\autocite[341]{taruskin1982} Taruskin states that
authenticity in ``performance practice is a chimaera, most of us are
nevertheless no more deterred by this realization from seeking it than
was Bellerophon himself.''\autocite[341]{taruskin1982} The essence of
Taruskin's thought on the subject in 1982 seems to be: ``it's find to
assemble the shards of a lost performance tradition, but how much
better to reinvent it.''\autocite[343]{taruskin1982} In speaking of
the most authoritative performances he asserts that they ``have always
been those that have proceeded from a vividly imagined---that is
frankly to say imaginary---but coherent performance style. They
provide themselves with Tradition \ldots and bestow authenticity upon
themselves.''\autocite[343]{taruskin1982} These performances are not
truly reconstructions of past performances, ``they are
quintessentially modern performances, modernist performances in fact,
the product of an esthetic wholly of our own era, no less time-bound
than the performance styles they would
supplant.''\autocite[344]{taruskin1982} At the end of this article,
Taruskin quotes an analogy for the musicologist from Nikolai Malko in
which the musicologist is like a cook who only talks about the making
of eggs instead of actually making them. After this Taruskin states
that in reference to those eggs, ``we're eating them now, and even
cook up a few on occasion, as when we do a little discreet composing
to make a fragmentary piece performance. Now, if we could only sell
them\ldots''\autocite[349]{taruskin1982} This is an extremely
interesting statement. It references a form of authenticity that Kivy
mentions later in 1995, the `sensible authenticity' in which the
audience decides their own form of authenticity.\autocite{kivy1995}

Laurence Dreyfus, in his 1983 article, ``Early Music Defended Against
its Devotees'', discusses Theodor Adorno's thoughts on early music
performance and the developing aesthetic relationship between the
emerging avant garde and early music. Dreyfus's article also attempts
to examine many of the emerging connections within early music. One
distinct way that he does this is through what he calls a Brechtian
table in which he defines the different roles of Early Music against
the Musical Mainstream.\autocite[317--318]{dreyfus1983} This table has
been printed below.
\clearpage
\begin{multicols}{2}
  Early Music\\
  \begin{enumerate}
  \item The conductor is banished.
  \item All members of the ensemble are equal.
  \item Ensemble members play a number of instruments, sometimes sing, and commonly exchange roles.
  \item Symptomatic grouping: the consort---like-minded members of a harmonious family.
  \item Virtuosity is not a set goal and is implicitly discouraged.
  \item Technical level of professionals is commonly mediocre.
  \item The audience (often amateurs) may play the same repertory at home.
  \item The audience identifies with the performers.
  \item Programs are packed with homogeneous works and are often dull.
  \item Critics report on the instruments, the composers, pieces and that ``a good time was had by all.''
    \end{enumerate}
    \columnbreak
    Musical Mainstream\\
    \begin{enumerate}
  \item The conductor is the symbol of authority, stature, and social difference.
  \item The orchestra is organized in a hierarchy.
  \item The ``division of labor'' is strictly defined, with one player per part.
  \item Symptomatic grouping: the concerto---opposing forces struggling for control; later, the one against the many.
  \item Virtuosity defines the professional.
  \item Technical standards are high and competitive.
  \item The audience marvels at the technical demands of the repertory.
  \item The audience idealizes the performers.
  \item Programs contain contrasting items and are designed around a climax.
  \item Critics comment on the performer and his interpretation.
  \end{enumerate}
\end{multicols}

In Taruskin's next article on the subject of authenticity, he notes
that performers see musical performance as texts. He notes that
``many, if not most, of use who concern ourselves with `authentic'
interpretation of music approach musical performance with the
attitudes of textual critics, and fail to make the fundamental
distinction between music as tones-in-motion and music as
notes-on-page.''\autocite[4]{taruskin1984} This discussion of
performer as textual critic is placed in opposition to the performer
as moral philosopher. I must add this excellent quote which has bearing on finger-style guitar:
\begin{quote}
  So where does one begin? Surely with the music, with one's love for
  it, with endless study of it, and with the determination to
  challenge one's every assumption about it, especially the
  assumptions we do not know we are making because, to quote
  Whitehead, `no other way of putting things has ever occurred' to us
  \ldots One Musician whom I particularly admire, a lutenist, once
  told me that when he began to experiment with improvisation
  practices to accompany medieval song, he deliberately
  mistuned[\emph{sic}] his instrument so that his fingers would not be
  able to run along familiar paths.\autocite[10]{taruskin1984}
\end{quote}
It is curious to think of the early music performer as sharing
characteristics with the 21st-century finger-style guitarist. In a
segment which Kivy might deem a discussion on `sonic authenticity,'
Taruskin notes that using old instruments forces the performer into a
space of unfamiliarity which forces her ``into a more direct confrontation with the music.''\autocite[11]{taruskin1984}

The next article, was written by Channan Willner in 1990. It focuses
on the then rising trend in the recording of Beethoven symphonies of
using what are thought of as period instruments. Of interest is the
statement that in order for an effect performance of Beethoven on
period instruments there would be a requirement of ``the leadership of
a historically informed conductor of the same high quality, one who
possessed commensurate experience (especially the once-traditional
long apprenticeship) and could draw the right kind of expressive
response from a group of period players.''\autocite[89]{willner1990}
This is quite interesting once one thinks of Taruskin's discussions on
how the performer of early music might also need to be a scholar of
early music. Here, Willner seems to be stating that a conductor would
need to be at the same level if not higher than the players in a
period orchestra in order for the performance to be effective. It is
also interesting to note that the performance of Beethoven symphonies
seems to go against Dreyfus's assertion that there are no conductors
in early music.

By all accounts it sounds as if the period instrument discussion was
being had in all sorts of musical circles. In \emph{The Musical Times}
in 1991, Leo Black states that the ``whole issue of `authenticity', of
`period-instrument' performance, has got out of hand, though so much
money is involved that it isn't going to go
away.''\autocite[64]{black1991} It is interesting to note that Black
goes on later in this brief article to espouse the music of Reinhard
Goebel who, to my ear, seems to have taken the `whole issue' to a
place that someone might find difficult. Perhaps Dreyfus's comparisons
of early music to the avant garde are useful here. Where a critics
voice seems to say that `the listener should listen to this obscure,
period-instrument, performance by a German conductor of some very
well-known music. You might not like it.'

A deeply difficult-to-read article was published in Noûs in 1991 by
Stephen Davies.\autocite{davies1991} If I had more time this semester
I feel that this article would reveal itself as a very curious
discussion of the performance of music from an ontological
perspective. This article begins with a brief recap of the individuals
who are writing about the performance practice debate and couches it
within a perspective of the nature of the musical work.

\nocite{sherr1991}

From my perspective it does not seem to have taken a long time for the
term `authenticity' to be defined in the manner that Lewis Lockwood
says that Taruskin defined it in 1988 in ``The Pastness of the
Present, and the Presence of the Past'' as `commercial
propaganda.'\autocites[137]{taruskin1988}[as quoted
in:][]{lockwood1991} The idea that authenticity is a marketable term,
especially within classical music, is certainly going to be a
potentially strong theme in the discussions of authenticity. I am
quite surprised that there wasn't more discussion about the
possibility that the push for authenticity in the performance of early
music is a push for capitalizing on an obscure component of classical
music. It would be interesting to read any analyses of authenticity
and Marxism or a reading of authenticity and capitalism.

Daniel Leech-Wilkinson's article in \emph{Companion to Medieval \&
  Renaissance Music} examines the different ways in which
contemporaneous sources talked about the production of music and
musical aesthetics as a way to engage with the trend of authenticity
in early music.\autocite{leech1992}

Dreyfus's article takes offense at the authenticity trend in early
music's approaches to the performance of Mozart.\autocite{dreyfus1992}
In the same issue of \emph{Early Music}, Taruskin describes the
authenticity debate as ``our ongoing War of the
Buffoons.''\autocite[311]{taruskin1992} Taruskin also sums up his
position by stating
\begin{quote}
  that the ancients and moderns ought to exchange labels. What is
  usually called `modern performance' is in fact an ancient style, and
  what is usually called `historically authentic performance' is in
  fact a modern style.\autocite[311]{taruskin1992}
\end{quote}
Taruskin's main point in 1992 seems to be that `historical'
performances are in fact representative of modern values. The
performance is modern because of its defamiliarization and its
distance from the oral traditions which pervade the performances of
Beethoven, Brahms and the like.\autocite[314]{taruskin1992}

At a symposium in Berkeley, Dreyfus places a hefty amount of blame on
the musicologists for creating the fragmentation that could be heard
in the performance of early music in the early 1990s. He states that
the
\begin{quote}
  famous debates and sometimes gratifyingly raucous polemics about
  overdotting, vibrato, and the performance of trills have shaped a
  generation of musicians who imagine the historicist enterprise as a
  sum of accrued details, useful for shaming colleagues into observing
  yet another prohibitive taboo.\autocite[116]{kerman1992}
\end{quote}
The idea seems to be that through detailed research and healthy debate
on performative aspects of early music, the scholars have affected the
performance of its music in a deleterious manner. Dreyfus goes on to
recommend that the antidote to this is for the performer to read the
philosophers and critics of the era in which the performer is
performing in order to attempt to understand the context of the music
a little better. He goes on to recommend explicitly that the performer
examine the concept of `the sublime.'\autocite[116]{kerman1992} The
article from which these quotes are taken would be an interesting read
in the future on how the voices of the scholar and the music critic
coalesce and diverge when discussing early music performance practices
from a historical sense and from a 20th-century context.

In a different, more organized, world, I would have read most of
\emph{Text and Act} by Richard Taruskin. This book contains such
wonderfully pithy statements which help situate Taruskin's views on
the subject. For instance, ``what we call historical performance is
the sound of now, not then. It derives its authenticity not from its
historical verisimilitude, but from its being for better or worse a
true mirror of late-twentieth-century
taste.''\autocite[166]{taruskin1995} This seems to be the kernel of
Taruskin's argument in the debate about authenticity. The authenticity
of the performance of early music is based upon its authenticity in
the present moment not in the past moment. Taruskin seems to believe
that this type of authenticity is better than the other type of
authenticity. It is a question of taste not a question of
authenticity for Taruskin.

Shai Burstyn gave an account of a symposium at the Jerusalem Music
Centre called \emph{Authenticity in
  interpretation}.\autocite{burstyn1995} This report gives a good
description of how players interact with the scholarship on the
subject. This is important because these individuals seem to differ in
opinion from Taruskin and Dreyfus.

The next major publication in the field is a review of Taruskin's
\emph{Text and Act} by John Butt.\autocite{butt1996} The article
starts with statement that this is a field that ``Taruskin has not
only dominated, but largely defined over the last 15
years.''\autocite[323]{butt1996} Butt provides the opinion that
Taruskin is actually on the side of historically informed performance
by saying that Taruskin only wanted to show the shortcomings of the
movement.\autocite[325]{butt1996} Butt attempts to show how Taruskin's
arguments are some times contrary to each other. Butt posits that
Taruskin's ``desire to `democratize' performance by catering to the
needs and wishes of the audience\ldots and his tendency to promote
postmodernism as the answer to all modernism's ills'' is a difficult
position to defend. The idea that the audience decides what is good
and what is bad is difficult to quantify. There is not some rubric
that the scholar is able to use to determine if the audience likes
something. Butt takes offense to the idea that postmodernism is able
to answer any of the questions that are posed by this or any other
debate. Butt see this as an attempt at utopianism which Taruskin
attempts to remove from his assessment of the postmodern turn of early
music performance.

Kivy ``proposes there are four'' authenticities.\autocite{kivy1995} As
discussed in Murphy, there is the `composer authenticity,' the `sonic
authenticity,' the `personal authenticity,' and the `sensible
authenticity.'\autocite[26]{murphy2008} The first deals with composer
intent, the second, sonic, is concerned with ``reconstructing the
physical materials'' surrounding a given composition's inception, the
third, personal, deals with the performer sense of authenticity, the
fourth, sensible, deals with how the audience interacts with
authenticity.

\section{Conclusions}
\label{sec:conclusions}

Like most of the work that I have done this semester, I find myself
biting off more than I can chew, as it were. This project demanded
more time than I was able to give. It is a worthwhile subject. In a
different world, I would have worked more with recent dissertations on
the performance of early music. It also would have been beneficial for
me to work more exclusively with dialogues in the 2000s as it seems
that this is when a plurality of voices were able to engage with the
debate started in the early 1980s by Taruskin and others. I also would
have been interested in focusing on thoughts on early music
performance surrounding plucked-string instruments, such as lutes and
historical guitars. It appears that these discussions have been
written about in journals such as \emph{Performance Practice Review}
and \emph{Early Music}.


\clearpage
\nocite{*}
\printbibliography
\end{document}
%%% Local Variables:
%%% mode: latex
%%% TeX-master: t
%%% End:

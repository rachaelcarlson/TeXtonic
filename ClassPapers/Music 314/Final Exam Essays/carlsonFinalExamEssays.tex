\documentclass[12pt]{article}

\usepackage{fontspec}
\setmainfont[Ligatures=TeX,Mapping=tex-text,Numbers=OldStyle]{Garamond Premier Pro}
\setmonofont{Garamond Premier Pro}

\usepackage[]{biblatex-chicago}
\addbibresource{references.bib}

\author{Rachael Carlson}
\title{Music 314: \\ Final Exam Take-Home Essays}
\date{\today}

\begin{document}
\maketitle

\section*{Essay 1}
\label{sec:question-1}

\emph{1a. Explain the concept of Postmodernism, as it contrasts with
  Modernism, and reflects changes in attitudes about the relationship
  between the past and the present. Detail some of the characteristic
  techniques of postmodern composition, giving specific examples from
  the repertoire.}

Postmodernism is marked by the ``building on the music of the past
without irony.''\autocite{week9} This is considerably different than
Modernism rejection of the past. Another way of looking at
Postmodernism is to see that it rejects and questions large narratives
in favor of many small narratives. These large narratives are
generally the those of the establishment. Postmodernism seems to
attempt to give voice to those who did not have voices in
Modernism. For instance, equal importance may be given to the
performer as it is to the audience. 

\section*{Essay 2}
\label{sec:essay-2}

\emph{2a. Referring to specific composers and works, discuss ways that different aspects of identity (race, gender, sexuality, etc.) have impacted the composition and reception of music in the late 20th and early 21st centuries.}

\end{document}
%%% Local Variables:
%%% mode: latex
%%% TeX-master: t
%%% End:

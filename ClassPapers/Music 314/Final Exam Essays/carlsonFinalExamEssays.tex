\documentclass[12pt]{article}

\usepackage{fontspec}
\setmainfont[Ligatures=TeX,Mapping=tex-text,Numbers=OldStyle]{Garamond Premier Pro}
\setmonofont{Garamond Premier Pro}

\usepackage[]{biblatex-chicago}
\addbibresource{references.bib}

\author{Rachael Carlson}
\title{Music 314: \\ Final Exam Take-Home Essays}
\date{\today}

\begin{document}
\maketitle

\section*{Essay 1}
\label{sec:question-1}

\emph{1a. Explain the concept of Postmodernism, as it contrasts with
  Modernism, and reflects changes in attitudes about the relationship
  between the past and the present. Detail some of the characteristic
  techniques of postmodern composition, giving specific examples from
  the repertoire.}

Postmodernism is marked by the ``building on the music of the past
without irony.''\autocite{week9} This is considerably different than
Modernism's rejection of the past. Another way of looking at
Postmodernism is to see that it rejects and questions large narratives
in favor of many small narratives. These large narratives are
generally the those of the establishment. Postmodernism seems to
attempt to give voice to those who did not have voices in
Modernism. For instance, equal importance may be given to the
performer as it is to the audience. The musical characteristics that
were explored in the class which are indicative of Postmodernism are
quotation, collage, polystylism, intertextuality, pastiche, etc.

A great contrasting example could be Mahler's Symphony No. 2 in C
Minor and Berio's \emph{Sinfonia} III. In this symphony, Mahler
attempts to build a world with a grand unified vision. In
\emph{Sinfonia} Berio constructs many small worlds through the use of
quotation. \emph{Sinfonia} also plays on the role of the audience. If
the audience understands the quotations then they will be within the
inner sanctum of listeners. If the listener does not understand the
quotation he or she will be unable to understand the composition. This
is certainly a different mentality toward the audience than that taken
during modernism in which the composer's used compositional techniques
which could easily be understand because they were techniques used
throughout Western Art Music's history.

One of my personal favorite techniques used in Postmodernism is
Pastiche, which is a playing on the expectations of music from the
past. Many of the examples given in the \emph{Grove} and elsewhere
give examples which we might consider neo-classicism, such as
Prokofiev's Classical Symphony and Stravinsky's
\emph{Pulcinella}. These, however, are examples of neo-classicism. The
composers are recalling certain stylistic elements of a
composition. In Schnittke's Concerto Grosso No. 1, the composer sets
up the expectation that it is going to be a neo-classical composition
but then seems to parody the form and content of the that which the
audience is expecting.

\section*{Essay 2}
\label{sec:essay-2}

\emph{2a. Referring to specific composers and works, discuss ways that different aspects of identity (race, gender, sexuality, etc.) have impacted the composition and reception of music in the late 20th and early 21st centuries.}

As noted above, Postmodernism gives voice to many narratives. Once
these narratives are given voice, they are able to have conversations
with one another. There are several excellent examples of individuals
different ways in which identity is involved with the composition and
reception of music in the late 20th and early 21st
centuries. Composers such as Lewis, Temple, and Du Yun all provide
interesting, thought-provoking voices in their music.

George Lewis, in an essay in \emph{Black Music Research Journal},
spoke of two different voices of improvisation in Western Art Music,
Eurological and Afrological.\autocite{lewis} The primary thought
behind these two different voices of improvisation is that one derives
from European forms of improvisation dating to 150 years prior to 1996
and the other derives from a strong tradition of improvisation in
Africa. Lewis draws upon both of these voices with his composition
\emph{Voyager} which utilizes an improvisatory computer program which
improvises with the performer.

In an blog article Alex Temple notes that she doesn't ``think of my
work as specifically female, [however] I \emph{do} think of it as
specifically genderqueer.''\autocite{temple} In her composition,
\emph{Behind the Wallpaper}, she notes that it is music of and about
being an outsider. She states, about this composition, that she
``wanted to create something that my fellow trans and/or genderqueer
poeple in particular could listen to and say `yes, I know what that's
like,' or maybe even `you mean I'm not the only who's experienced
that?'''\autocite{temple}

I think that Du Yun's \emph{Angel's Bone} is an incredibly wonderful
example for our identity can change how music is composed. In a genre
of music such as Opera which contains such hits as \emph{Don
  Giovanni}, it is refreshing to hear about an opera which deals with
real world issues and does not glorify womanizing white men.

\printbibliography
\end{document}
%%% Local Variables:
%%% mode: latex
%%% TeX-master: t
%%% End:

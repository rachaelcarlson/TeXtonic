% Created 2017-04-19 Wed 00:51
\documentclass[11pt]{article}
\usepackage{fontspec}
\usepackage{graphicx}
\usepackage{longtable}
\usepackage{float}
\usepackage{wrapfig}
\usepackage{rotating}
\usepackage[normalem]{ulem}
\usepackage{amsmath}
\usepackage{textcomp}
\usepackage{marvosym}
\usepackage{wasysym}
\usepackage{amssymb}
\usepackage{hyperref}
\tolerance=1000
\author{Rachael Carlson}
\date{\today}
\title{outline}
\hypersetup{
  pdfkeywords={},
  pdfsubject={},
  pdfcreator={Emacs 25.2.1 (Org mode 8.2.10)}}
\begin{document}

\maketitle
\tableofcontents

\section{Outline}
\label{sec-1}
\subsection{Purpose}
\label{sec-1-1}
\begin{itemize}
\item The purpose of this project is to attempt to ready myself for life outside
the university. One of the goals of this project is to help me spend as
little time outside the university as possible. In producing this paper, I
attempt to anticipate[slide] the components of a proposal for a
finger-style guitar program which would most likely entice or convince an
institution of higher education of the importance of offering such a
program. I then prepare[slide] for as many anticipated components as
possible. At the moment I see this as entailing three different
tactics[slide]: situating[slide] the guitar within its contexts;
projecting[slide] into the future (think anticipation); and
professional[slide] development of myself as an instructor.
\end{itemize}
\subsection{Tactics}
\label{sec-1-2}
\begin{itemize}
\item Situation: placing finger-style guitar within its contexts both
contemporary and historical.

\item Projection: attempting to anticipate changes in finger-style guitar
instruction, performance, and composition as a way in which to
excite. Also, a projection into the benefits that a university may
experience with the inclusion of a finger-style program.

\item Personal Growth: Curriculum Vitae
\end{itemize}
\subsubsection{Situation}
\label{sec-1-2-1}
\begin{enumerate}
\item Contextualization
\label{sec-1-2-1-1}
\begin{itemize}
\item Situate finger-style as a manner of playing the instrument within different
contexts[slide]. Historical: race and hillbilly records, Ragtime, parlor
guitar. Regional: Hawaiian, Russian, Venezuala/Argentina/South America,
British, French\ldots{} By Genre: Rock, hip hop, indie, finger-style as a
genre\ldots{}
\end{itemize}
\item Historiography
\label{sec-1-2-1-2}
\begin{itemize}
\item Be prepared to answer all manner of questions relating to the history of
finger-style guitar and the guitar in general. Due to the nature of playing
the instrument, be prepared to answer major questions about the
lute---mostly relating to the renaissance and baroque lutes.

\item Most importantly, known the history of finger-style guitar as a genre. This
cannot but be a biased account of the history. I think that we must embrace
this bias and give the best account possible knowing that we can change our
understanding of history with new evidence.
\end{itemize}
\end{enumerate}
\subsubsection{Projection}
\label{sec-1-2-2}
\begin{enumerate}
\item Future Developments
\label{sec-1-2-2-1}
\begin{itemize}
\item Anticipate[slide], as thoroughly as possible, the potential avenues in
which finger-style guitar may travel. Anticipate its popularity by
reflecting on how it has been growing. If possible, use qualitative
information to do this.

\item Other disciplines within the musical academy seem to have canonized their
repertoire. Is this desirable? Should the music of finger-style guitar as a
genre be canonized? Or is it more helpful for the music to resist
canonization? Is it better to be inclusive or exclusive?
\begin{itemize}
\item One interesting area to think about would be the difference in repertoire
from 2006 to 2017. The amount of repertoire available to the student of
finger-style guitar in 2006 was much less than the student of 2017. Is
this a good thing? I am reminded of \emph{The Paradox of Choice: Why More is
Less} (2004) by American psychologist Barry Schwartz.
\end{itemize}
\end{itemize}

\item Potential Gains
\label{sec-1-2-2-2}
\begin{itemize}
\item Think about different ways in which the offering a finger-style guitar track
in a guitar program could benefit a university.
\item From a monetary point of view, it is possible, due to the rising popularity
of finger-style guitar as a genre, that a school, with the right backing,
marketing, and encouragement, could generate a hefty amount of cash from
bringing in students from all of the world to study at their university.
\item A more practical gain, and certainly one which I would find more interest,
is in the scholarly gains which could be had from offering a program. I can
see it now, a university which able to establish the relationship between
the parlor guitar, the Sears-Roebuck Catalog, and the rise of folk music in
the United States through the thread of finger-style guitar.
\begin{itemize}
\item I can see a peer-reviewed journal published by a university guitar
program, a feat which I am unaware of in general let alone for a
finger-style program.
\end{itemize}
\item Can you think of other ways in which a university might benefit from having
a finger-style guitar track?
\end{itemize}
\end{enumerate}
\subsubsection{Personal Growth}
\label{sec-1-2-3}
\begin{enumerate}
\item Compositions
\label{sec-1-2-3-1}
\begin{itemize}
\item The production of an album of music may help to establish my credibility.
\item The actual composition of music seems to set the finger-style guitarist
apart from his or her classical, nylon-string guitar counterpart
\end{itemize}
\item Performances
\label{sec-1-2-3-2}
\begin{itemize}
\item Prestigious performances all of the globe
\end{itemize}
\item Publications
\label{sec-1-2-3-3}
\begin{itemize}
\item Magazines, Music Blogs, Scholarly Journals, Transcriptions and other Sheet
Music, books
\end{itemize}
\item Other Ideas?
\label{sec-1-2-3-4}
\begin{itemize}
\item Can you think of other ways in which a finger-style guitarist could help
establish his or her credibility in the eyes of a university
\end{itemize}
\end{enumerate}

\subsection{Institutions}
\label{sec-1-3}
\begin{itemize}
\item This list is far too short. There must be others out there that teach
finger-style guitar at the collegiate level. Some of them, like a few I am
about to mention, probably offer finger-style lessons but do not advertise
as such.
\item While I wanted to add Antoine Dufour and Adam Cord to this list, I was
unable to determine the institution with which they were respectively
associated.
\item Michael Chapdelaine teaches classical and finger-style guitar at the
University of New Mexico. In an interview, he stated that he blurs the
lines between classical and finger-style. Once a student has reached a
certain level he or she is able to study finger-style guitar.
\item Sean McGowan teaching finger-style jazz and commercial guitar at the
University of Colorado-Denver. I look forward to speaking with him about
his pedagogical approaches. Alex de Grassi stated in an interview that
McGowan offers a class on finger-style guitar which uses both de Grassi and
McGowen's respective finger-style methods. McGowan's being a jazz
finger-style method.
\item Michael Millham teaches classical guitar at Eastern Washington
University. Will Boulé and Alex de Grassi indicates that he offers
finger-style lessons.
\item John Stropes is no stranger to anyone here. It is worth noting that John
Stropes has a long history in guitar instruction and publication. The UWM
program was initially formed through a joint venture with the Wisconsin
Conservatory of Music and UWM. John can you speak a little about the
creation of this program?
\end{itemize}

\subsection{Interviews}
\label{sec-1-4}
\begin{itemize}
\item I plan to interview the people that I have just mentioned in order to
determine out how they got where they are and how I can get there.
\item I am also looking for recommendations from these individuals for a course
of action.
\end{itemize}

\subsection{Elevators}
\label{sec-1-5}
\begin{itemize}
\item I do not currently have an elevator speech for this paper. I anticipate
that this will be the most difficult aspect of this work. It will be like a
puzzle in the removal and movement of text will result in a different
picture.
\end{itemize}

\subsection{Conclusion}
\label{sec-1-6}
\begin{itemize}
\item In conclusion, this paper is designed to be as practical as possible. In
doing so, it has become massive. I will not be able to get to everything
that I have talked about. I do not anticipate that this will be an easy
process. Like those trail blazers before me, I hope to find a path which
will lead toward the benefit of finger-style guitar and the
self-sufficiency that is a hallmark of success.
\end{itemize}

Thank you.
% Emacs 25.2.1 (Org mode 8.2.10)
\end{document}
%%% Local Variables:
%%% mode: latex
%%% TeX-master: t
%%% End:

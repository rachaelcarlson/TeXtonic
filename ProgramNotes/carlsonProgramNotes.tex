\documentclass{tufte-handout}
%\usepackage[margin=1.5in]{geometry}
\usepackage{fontspec}
\usepackage{ifxetex}

\setmainfont[Ligatures=TeX,Numbers=OldStyle]{Adobe Garamond Pro}
\setsansfont{Adobe Garamond Pro}

\ifxetex
  \newcommand{\textls}[2][5]{%
    \begingroup\addfontfeatures{LetterSpace=#1}#2\endgroup
  }
  \renewcommand{\allcapsspacing}[1]{\textls[15]{#1}}
  \renewcommand{\smallcapsspacing}[1]{\textls[10]{#1}}
  \renewcommand{\allcaps}[1]{\textls[15]{\MakeTextUppercase{#1}}}
  \renewcommand{\smallcaps}[1]{\smallcapsspacing{\scshape\MakeTextLowercase{#1}}}
  \renewcommand{\textsc}[1]{\smallcapsspacing{\textsmallcaps{#1}}}
\fi
\begin{document}
\begin{fullwidth}
\begin{center}
  Program\\
  Master's Recital in Finger-Style Guitar\\
  Rachael Carlson\\
  Greene Hall\\
  April 22, 2017, 7:30pm
\end{center}
\noindent Black Moon/Westward Move (2004)\dotfill Clive Carroll\\
\strut \hfill (b. 1975)\\

\noindent Partita No. 3 in E Major: III. Gavotte en Rondeau (1720, 1736)\dotfill Johann Sebastian Bach\\
\strut \hfill (1685 -- 1750)\\
\strut\hfill arranged by Rachael Carlson (b. 1988)\\

\noindent Watch the Tiger Walk (2016)\dotfill Rachael Carlson\\
\strut\hfill (b. 1988)\\

\noindent Little Meggie (2001)\dotfill Michael Gulezian\\
\strut \hfill (b. 1957)\\

\noindent Ants (c. 2008)\dotfill Leo Kottke\\
\strut\hfill (b. 1945)\\

\noindent The Staten Island Ferry (2017)\dotfill Rachael Carlson\\
\strut\hfill (b. 1988)\\

\noindent Studio N. 5 (2011)\dotfill Pino Forastiere\\
\strut\hfill (b. 1966)\\

\begin{center}
  \emph{Intermission}
\end{center}

\noindent Guitar Chimes (1929)\dotfill Arthur ``Blind'' Blake\\
\strut\hfill (1896 -- 1934)\\

\noindent Six-String Suite: Eggtooth/Grim to the Brim/Orange Room (1982)\dotfill Leo Kottke\\
\strut\hfill (b. 1945)\\

\noindent Madness (2013)\dotfill Matt Bellamy\\
\strut\hfill (b. 1978)\\
\strut\hfill arranged by Luca Stricagnoli (b. 1991)\\

\noindent The Happy Couple (1981)\dotfill Michael Hedges\\
\strut\hfill (1953 -- 1997)\\

\noindent Aerial Boundaries (1984)\dotfill Michael Hedges\\
\strut\hfill (1953 -- 1997)\\
\end{fullwidth}

\clearpage

\section*{Program Notes}
\label{sec:program-notes}

\subsection*{``Black Moon''/``Westward Move'' by Clive Carroll}
\label{sec:black-moonw-move}
The inspiration for this composition came from the realization that after every break-up with a significant other the composer found himself moving westward in the city of London, England. This composition is a representation of this realization.

An intriguing component of this composition is its treatment of melody. In ``Black Moon'' the melody is presented in a slow, stark, and free manner. This presentation allows the listener to hear the main melody, presented in Appendix 1, and potentially recognize this melody when it appears again later in the composition. This re-presentation of the melody occurs at the climax of the composition. 

``Black Moon'' poses a challenge for the performer. It is to be performed ``slow, freely'' as indicated by the composer's score. The challenge for the performer is to maintain the melodic lines while meaningfully interacting with the rubato. The performer must also present the ``Black Moon'' melody with the knowledge that it will be reiterated later in the composition. When the melody from ``Black Moon'' is reiterated in ``Westward Move'' the performer is confronted with the inherent difficulties of performing music with three voices along with the complication of reminding the listener that the melody from ``Black Moon'' is integrated into ``Westward Move.''

\subsection*{Partita No. 3 in E Major: III. Gavotte en Rondeau by Johann Sebastian Bach, arranged by Rachael Carlson}
\label{sec:partita-no.-3}
% \noindent Arranging the compositions by J. S. Bach for the six-string guitar has had an interesting history. Francisco Tárrega transcribed the fugue from Violin Sonata No. 1, \textsc{bwv} 1001. Andrés Segovia made the practice of transcribing Bach quite popular. The reception of Segovia's arrangements of Bach appears to be one of near-unanimous appreciation in Paris and London between the years of 1924 and 1935. One review, from \emph{The Musical Times} in 1935, stands apart from the rest as particularly effusive:
% \begin{quote}
%   a hasty judgment would assume the genuine guitar pieces to have been the successes and the Bach-Handel transcriptions a mistake. Actually, through the genius of Segovia, they were revelations. The Chaconne was what Milton must have meant when he talked about an `undisturbèd song of pure concent.' To hear Segovia achieve these miracles with an instrument so limited in its resources as the guitar is an extraordinary lesson in the subtleties of art. He must be heard to be appreciated, but as an example of his skill one may note that though the guitar's range of emphasis is small (for the strong accents of the violin lie beyond its capacity) Segovia contrives his perfect melodic and rhythmic phrasing by supplementing emphasis with spacing.
% \end{quote}

In preparing the arrangement of this movement, I used two different
manuscripts as sources. There are two extant versions of this composition. The
first comes from a manuscript dated 1720. The second comes from a manuscript
dated circa 1736. The first is orchestrated for solo violin without basso
continuo, without bass or accompaniment. The second arrangement is
orchestrated for a solo keyboard instrument. The arrangement from 1736 has
been the source of much confusion in the plucked fretted-string instrument
world. At its first discovery, it was thought that the arrangement was for
lute. Some of the latest research shows that it might have been orchestrated
for \emph{Lautenwerk}, a type of harpsichord which uses gut strings, or some
other keyboard instrument, not the lute as previously believed. Other
researchers assume that the \emph{Lautenwerke} are a set of pieces that
J. S. Bach wrote for the baroque lute. Engaging with this conversation, Peter
Williams states in a recent biography of J. S. Bach that he ``had a
\emph{Lautenwerk} made in Cöthen and again in Leipzig'' he also had one made
by his cousin Johann Nicolaus Bach of Jena in 1715 in Weimar. Williams
highlights a curious component of the 1736 arrangement of \emph{bwv} 1006: the
dynamic markings from the 1720 manuscript are kept in this arrangement. This
is curious because the \emph{Lautenwerk} is described as producing a dynamic
range of ``soft to very soft.'' The ability of this instrument to produce a
\emph{forte} dynamic is questionable. The debate over which instrument Bach
composed the 1736 arrangement continues.

\subsection*{``Watch the Tiger Walk'' by Rachael Carlson}
\label{sec:watch-tiger-walk}
The primary thought in this composition is the manner in which a tiger's shoulder blades move up and down while it is walking; it is deliberate, relaxed, and with conviction. Each placement of a paw on the ground seems to be done skillfully and with careful thought. This composition is a meditation on intention.

This piece is written in B minor. The meter is mostly in 4/4 with an
occasional measure of 6/4. The introductory material of this composition
establishes the primary rhythmic motive which is played on the \emph{and} of
beat 2, the downbeat and the \emph{and} of beat 3, and the downbeat of beat
4. This rhythmic motive is repeated throughout. It reappears
several times in the introductory material in several different iterations in an
effort to establish its presence throughout the piece. For instance, in m. 3,
beat 3 is performed with two different types of percussion. On the downbeat of
beat 3 the percussion is performed by bringing the thumb of the right hand
down on strings 5 and 6 in order to produce a percussive effect that sounds
similar to a snare drum. On the \emph{and} of beat 3, the percussion is
performed by the combination of left- and right-hand actions. The right-hand
thumb is damping strings 5 and 6. The left-hand first finger plays an
ascending-slur-from-nowhere on the seventh fret. The combination of the
right-hand damping of the strings and the left-hand the slur produces a snare sound that is quieter and more distant than the direct, up-front sound of the snare on the downbeat of beat 3.

\subsection*{``Little Meggie'' by Michael Gulezian}
\label{sec:little-megg-rach}
The genesis of this composition came from a concert the composer gave in a private home in which the five-year-old daughter of the homeowners danced in an peculiar manner. It seemed to the composer that the dancer was dancing in a syncopated, scuttling manner. Gulezian found this style of dancing and the five year old a locus of inspiration for this composition. The inspiration from the syncopated dancing can be seen in mm. 19–22. The syncopation can be heard in m. 19 in the highest-sounding voice which plays on the third sixteenth note of beat 1, the first and second sixteenth notes of beat 2, the first, third and fourth sixteenth notes of beat 3, and the first, third, and fourth sixteenth notes of beat 4. On the \emph{and} of beat 4 in m. 20, the composer wrote a grace note glissando from A₃ to E₃ which is held for a complete duration of a dotted eighth note. This creates an interesting pocket of rhythm where the performer is holding a note over a barline into m. 21. The act of producing an attack on the first beat of a measure represents the scuttling of the five year old dancer.

Another interesting technical component of the performance ``Little Meggie'' is the glissando from the G₃ to D₄ on the \emph{and} of beat four of m. 36. The execution of this glissando is challenging (and fun) due to the how the D₄ is executed—it is a harmonic. The performer must execute this maneuver with a certain degree of finesse in order to have the termination of the glissando result in the production of a natural harmonic.

  The coda in this composition requires attention from the performer, as well. One of the major challenges of performing this coda is that its contents is quite different from the rest of the composition. It starts with quick rolled harmonic chords which anticipate the final chord. The coda ends with the highest pitched note in the composition, a D₆. The performer must ensure that the music which precedes this coda does not interfere with the ecstatic nature of this coda. The preceding music must lead the listener to this revelry at the composition's conclusion. 

\subsection*{``Ants'' by Leo Kottke}
\label{sec:ants-leo-kottke}
Kottke has not released ``Ants'' on an album, though he has performed it in concerts since 2006. Like ``Gewerbegebiet'' by Leo Kottke, this composition is written in D minor with the guitar tuned in G major. This poses a challenge for the performer. The second string is tuned to B₃. This string makes it easy for the performer to accidentally sound the second string. This would potentially ruin the harmonic function of the composition. 

There are 224 measures in this composition. An analysis of the form would reveal that the ending or coda begins in m. 113, which is roughly the middle of the composition. The formal structure of this composition poses a challenge to the performer. There are not a large number of compositions which the end of the composition halfway through. The performer needs to ensure that listener feels that the composition is starting its ending at m. 113. The intended feeling would not be dissimilar from a roller coaster ride which seems like it should be out of momentum and yet it continues to thrill and exhilarate the rider.

There is a melodic motive in this composition which seems to elicit a sense of ants marching. This motive is first presented in mm. 3--11. This motive is written in alternating measures of 6/4 and 5/4. At the beginning of each measure of this motive is a B♭₃ which descends down to an A₃ then to a G₃. This motive reappears throughout the composition in mm. 50--54, mm. 90--92, mm. 113--124, m. 151, mm. 189--190, and in the finale of the composition in mm. 219--221 where it ascends to B♭₃ and then descends to D₂. 
\subsection*{``The Staten Island Ferry'' by Rachael Carlson}
This composition started as an homage to Hawaiian slack-key guitarist and composer Keola Beamer, however, instead of making one think of a serene Hawaiian beach, the intent is to remind one of a jostling ferry ride to and/or from Staten Island in New York City. Instead of the smell of sea salt and prehistoric volcanic ash, one smells the distinct and inimitable garbage of New York City in the sweltering summer heat. While the listener is waiting to board the Staten Island Ferry she sees someone in Bermuda shorts and a Hawaiian shirt, visibly upset by the heat. The listener is then packed into a ferry which seems to be older than the islands from which it departs and arrives. The ferry is a smooth enough ride. The listener thinks that she could do this on a daily basis. Suddenly the first wave rocks the ferry. Then the listener sits back in the hard wooden seating and looks for something tied down in order to steady herself. After a moderately uncomfortable ride she looks out the window of the ferry and sees the Statue of Liberty seemingly enjoying the fright that the listener is experiencing. The thought which keeps the listener going through this experience is the delectable pizza and beer that is awaiting at the end of the Staten Island Ferry.

The jostling nature of the ferry ride is conveyed in this composition through a sense of shifting meter. The composition is played with a swing, meaning that two beamed eighth notes equal, roughly, a triplet quarter note with an eighth note. The introduction is placed in 7/4 with a feeling that the downbeat of the measure is on the second beat. The main theme arrives with a measure of 2/4 which act as a pickup into two measures of 3/4. The next measure acts as a pickup into three measures of 4/4 and a measure of 3/4. Just when the listener is beginning to think that she understands the meter of the piece it shifts; there is an unexpected wave which produces a bump in the journey.

\subsection*{``Studio N. 5'' by Pino Forastiere}

Released in 2011 on the album \emph{From 1 to 8}, a collection of etudes for finger-style guitar, this composition explores the coalescence of meter and melody. In a conversation with the composer, it was noted that this composition could be considered to be a rondo form. This first subject in the rondo is written in a 7/4. The notation is unclear in this regard. If notation was rewritten, it may be clearer if it was written in 7/8 with the eighth-notes beamed in groups of 2, 2, and 3. It has a mechanical, almost robotic feel. The second subject of the rondo, beginning on the \emph{and} of beat 7 of m. 6 and ending in m. 10, written in 6/4, demonstrates a flowing, melodic sensibility while the mechanical component of the previous subject continues. This flowing subject is then played an octave higher with a different intervallic relationship and the same rhythmic placement beginning on the \emph{and} of beat six of m. 10 and ending in m. 14. The composition explores the relationship between the mechanistic 7/8 and the flowing 6/8. 

\subsection*{``Guitar Chimes'' by Arthur ``Blind'' Blake}
This composition represents an often overlooked aspect of finger-style guitar. The first wave of finger-style guitar players do not receive a large (enough) amount of attention from contemporary finger-style guitarists and students of the instrument. There are not a large number of guitar solos from this era of recorded music. The guitar solos of Blake and Lonnie Johnson stand out as distinctive, daring compositions which hold the ear of the listener who may or may not be patiently waiting for a singer to begin singing. Some of these early guitar solos are quite virtuosic such as ``Got the Blues for the West End'' by Lonnie Johnson or ``Blind Arthur's Breakdown'' by Arthur ``Blind'' Blake. Blake's guitar solos, in particular, demonstrate an uncanny ability to sound like at least two guitars. He was billed during his time as sounding like a piano.

``Guitar Chimes'' was recorded in Chicago, Illinois in 1929 for Paramount Records. It is one of Blake's slower compositions. The slower tempo allows this composition to be more accessible for a new performer of this style of playing. This composition follows a fairly regular harmonic structure with a few harmonic substitutions. The focus of the left hand is to place the fingers in form of the chord that is either being played or implied with the right hand in order to ensure that the harmonic structure is not dismantled through incorrect harmonies. The thumb of the right hand performs a brushing motion on beats 2 and 4, generally. This brushing motion reinforces the harmonic structure. On beats 1 and 3 the thumb plays a single bass note. This is a foundational component of finger-style guitar. This convention is present in the music of John Hurt and Gary Davis. Blake does not completely conform to this convention---one of the reasons why Blake's music is so entrancing. Blake subverts this convention in several different manners. One manner in which Blake subverts this convention is by having the thumb of the right hand perform a bass note on the \emph{and} of beat 4 of some measures. These bass notes propel the composition forward. This composition contains several of the same idiomatic compositional motives from other compositions by Blake in the key of C.
\subsection*{Six-String Suite: ``Eggtooth''/``Grim to the Brim''/``Orange Room'' by Leo Kottke}
Kottke has performed these three compositions as separate compositions for many years. There was a short period of time in the early 1980s when Kottke was presenting these three pieces as a medley. It is this presentation that I am including in my recital. In a concert in 1982, Kottke notes that he wrote this
\begin{quote}
over the period of many years, actually unbeknownst to me, instead of being three separate and distinct pieces, this is actually one piece with three movements. I just discovered that they were all there without predetermining that they would be. It was a happy circumstance for me. We can all pretend that I've written something longer than three and a half minutes.
\end{quote}
The first movement is now known as ``Eggtooth.'' Kottke sometimes calls it ``A Little Snow Starts to Fall.'' It was written with Michael Johnson for the short film, \emph{A Little Snow Starts to Fall Again}. It was a duet with Johnson. ``Grim to the Brim'' was written for Kottke's father. In a concert, Kottke talked about how he frequently had to move as a child. In Cheyenne, Wyoming, he would get beaten up by his peers. He went to his father, who taught jujitsu, and asked ``would you teach me how to defend myself? And he took me to the backyard—took a mattress to the backyard and beat me up. What a fine man.'' It is difficult to convey the humor with which Kottke speaks about his father. The third movement was written for Kottke's son. In an introduction to this piece Kottke stated that he was sitting around with is son one day:
\begin{quote}
  thinking in the living room, waiting for something to happen, you know. And he said, ``gee Dad, some day I want to…'' and I thought he was going to say ``I want to be a fireman or a policeman or a guitar player.'' He says ``some day, Dad, I want to drink a whole can of beer.'' What a kid. I've seen him staring wistfully at the six-packs in the kitchen. I don't know if I should be concerned or not. Somebody asked him what he wanted to do when he grew up and he said ``nothing, like my Dad.'' Smart kid.
\end{quote}

``Eggtooth'' has two main sections. The first is in the key of G major and E minor. The second section is in A major. The first section is mostly in 3/4. The second section is in 4/4. These two sections do not bear any similarities to each other, yet, somehow they work perfectly together. The first section leads into the second section by way of a v-I cadence. In some live recordings of ``Eggtooth'' one can hear the audiences' reaction to this transition. They seem to exclaim or shout in excitement or anticipation.

``Grim to the Brim'' is in the key of B minor. It is comprised of four different sections. The first section, \textsc{a}, is played three times. The second section, \textsc{b}, is played once. The form of this movement is similar to a rondo in that the \textsc{a} section material is played several times. A formal analysis might reveal that the structure is similar to \textsc{a b c a d c a}.

``Orange Room'' is in the key of D major. It starts with an alternating bass between D₃ and A₂. This alternating bass then moves to A₂ and E₂ with an A minor harmony. The alternating bass pattern in the introductory section establishes the tempo while establishing the melody in the lower strings. Unbeknownst to the listener, the alternating bass line \emph{is} the melody in both of those sections. This movement serves as a spellbinding close to the medley.
\subsection*{``Madness'' by Matt Bellamy, arranged by Luca Stricagnoli}
The impetus to perform this arrangement is centered around innovations in finger-style guitar notation by John Stropes and Raj Chaudhuri. This new form of notation is being called grid notation after the visual component of the display of its information. This arrangement is well-suited for this new type of notation due to several factors in its execution. First, the arrangement uses a partial capo on the sixth fret on strings 5 through 1. This poses a challenge to the notation of the music. A partial capo makes it near impossible to read standard notation; one would not know whether the pitches indicated are in front of or behind the capo. Second, the percussion on the body and strings of the guitar are not easily notated in finger-style guitar notation. There has not been any established procedures for the notation of percussion on the guitar. Frequently, the composer or engraver of finger-style guitar music which utilizes percussion create their own system for each composition. This creates problems for students of these compositions as the student needs to learn a new system of percussion notation for each composer.

Grid notation is a successor to stringed-instrument tablature; it tells the reader where to place her fingers. The display of rhythm in tablature has changed over the course of the last six centuries. It started by writing the longest value in the composition as a quarter note and then each shorter time value is displayed in relation to the quarter note. These rhythmic values are displayed above the staff. Once the rhythmic value is displayed each subsequent note is of that value unless otherwise indicated.  The bulk of contemporary tablature does not list rhythmic values. Innovations in finger-style guitar notation have lead to displaying rhythmic values of the right hand below the staff. Grid notation displays rhythm differently than all of these previous methods. There are two measures per system. Each measure is divided into four beats, notated with vertical gray lines. The first beat of each measure is notated with a thick gray line which also serves as the barline. Each beat is subdivided into four sixteenth notes notated with a thin gray line. The numbers, which indicate the fret upon which the finger is to be placed, are typeset on the vertical gray lines.

The removal of traditional barlines from the notation allows for the synchronization of the score with a recorded audio/video performance. Barlines in traditional notation take up space in the notation but do not occupy time. This makes the synchronization of traditional notation and audio/video performances a nontrivial activity. Even when one is able to synchronize the audio/video to the notation, the beginning of every measure starts with a jump in the synchronization. Grid notation removes this issue by redefining the barline as the first beat of each measure.

Stropes and Chaudhuri have taken advantage of this redefinition of the barline by syncing the video Stricagnoli posted on YouTube to the grid notation. This was done as a means to facilitate the instruction of the performance of this arrangement. When the student is learning the arrangement if she has a question about a certain note in the piece she is able to go directly to the point in the video by referring to the synced audio/video/notation presentation. The ability to sync notation with audio/video sources represents a fresh, new direction in the notation of finger-style guitar.
\subsection*{``The Happy Couple'' by Michael Hedges}
\emph{Klangfarbenmelodie} is a term coined by Arnold Schoenberg in \emph{Harmonielehre} (1911) in which it is discussed that one can use timbre as a structural element of composition similar to pitch in the create of melodic motives. Arnold Schoenberg states:
\begin{quote}
  tone becomes perceptible by virtue of tone color, of which one dimension is pitch. Tone color is, thus, the main topic, pitch a subdivision. Pitch is nothing else but tone color measured in one direction. Now, if it is possible to create patterns out of tone colors that are differentiated according to pitch, patterns we call `melodies,' progressions, whose coherence (\emph{Zusammenhang}) evokes an effect analogous to thought processes, then it must also be possible to make such progressions out of the tone colors of the other dimension, out of that which we call simply `tone color,' progressions whose relations with one another work with a kind of logic entirely equivalent to that logic which satisfies us in the melody of pitches.
\end{quote}
This composition explores this technique through the use of two different strings playing the same pitch in a melody. The ability of the guitar to play a single pitch on several different strings usually poses issues for the performer and the composer. In order for the composer to ensure that a performer is going to play a pitch on a certain string of the guitar she must tell the performer upon which string she must play through the use of editorial methods. In standard notation, the convention of a circled number is utilized to convey this information: ①, ②, ③, etc. Tablature inherently solves this problem by telling the performer exactly where to place her fingers. In ``The Happy Couple,'' Michael Hedges emphasizes this unique capacity of the guitar to explore the concept of \emph{klangfarbenmelodie}. Hedges also utilizes a unique characteristic of the steel-string guitar, the ability to alter the tuning of the strings. The standard tuning of the guitar is E₂ A₂ D₃ G₃ B₃ E₄. Hedges changes the sixth-string, E₂, down to G₁. Steel-string guitar is capable of reproducing such a low note through a combination of the high tension of steel strings and the integration of electronic sound reinforcement intrinsic to the performance of finger-style guitar. It is also interesting to note that the interval between the fourth and third strings, E₃ and F♯₃ respectively, is two half-steps and the interval between the third and second strings, F♯₃ and A₃ respectively, is three half-steps. The proximity of these strings to each other helps facilitate the ability to play the same pitch on different strings.

The first subject in this composition presents a descending \emph{klangfarbenmelodie}, moving from E₄-C♯₄-A₃-F♯₃. It is worth noting in mm. 3 and 7 the A₃ played on three different strings thus allowing the performer to explore the concept of \emph{klangfarbenmelodie} in three different ways in the same measure. The material in mm. 1–8 is repeated again in mm. 45–48 and again in mm. 73–80. When this material is repeated in mm. 73–80, it is presented with a sixteenth-note sextuplet right-hand arpeggio figure.

Contrasting material to the \emph{klangfarbenmelodie} in mm. 1–8 is presented in mm. 9–12. This is a descending melodic motive of F♯₄-E₄-D₄-C♯₄-D₄-A₃-B₄-F♯₃. This motive is notably reiterated in mm. 81–84 and the first four pitches in mm. 93–94. The reiteration in mm. 81–84 continues the sixteenth-note sextuplets from mm. 73–80. This melody is played with the \emph{a} finger. In mm. 93–94, the conclusion of the composition, this contrasting material is presented in natural harmonics. This composition offers the performer a plethora of interpretive decisions. The performer is able to perform the \emph{klangfarbenmelodie} while exploring the spectrum of possibilities from making the \emph{klangfarbenmelodie}  as similar as possible or as dissimilar to each other pitch.
\subsection*{``Aerial Boundaries'' by Michael Hedges}
This composition was released on the album of the same name \emph{Aerial Boundaries}. In an interview with \emph{Fingerstyle Guitar} magazine, Hedges states \emph{Aerial Boundaries} ``has been my best selling record.'' In this same interview, Hedges discusses the process of learning to play this composition:
\begin{quote}
  In order to learn ``Aerial Boundaries'' properly, I had to write it all out in notes, like in piano notation. But I didn't write it down while I was doing it. That's just what was inside me. But looking at the notes, at the actual pitches, the standard notation with treble and bass clef enabled me to think more about the actual sound as I was doing it. It was just a training exercise that helped me when I was recording it.
\end{quote}
This composition stands as one of the pinnacles of finger-style guitar performance and composition.

A crucial element of the performance of this composition is the use of right-hand string-stopping. This technique was perfected by Hedges and is displayed in many of his compositions. It imparts an ``elegant sound'' and the performer must have ``perfect control over the duration of each note.'' Right-hand string-stopping came about for Hedges through the need of ``the composer makes a conscious decision not only when a note should start, but whether or not it should be stopped before natural decay would render it inaudible.'' Right-hand string-stopping is notated through the use of non-opaque lines in the tablature, either gray or red. These lines indicate that your ``right-hand finger should be resting on the string from the time the gray begins until it ends.'' John Stropes wrote about this technique in 1991 in \emph{Acoustic Guitar}:
\begin{quote}
  In ``Aerial Boundaries,'' right-hand string[-]stopping serves two functions. It terminates the sound of notes, and it also mutes the strings to provide a rhythmic tick when the left hand hammers on or pulls off. In measures 1 and 2…right-hand string[-]stopping helps define a line that is being articulated entirely by the left hand. (This left-hand rhythm, called \emph{masmudi sagheer}, is a popular Arabic rhythm.)
\end{quote}
The Arabic rhythm can be seen in the standard notation with the E₄ and A₃ pitches on the first and third sixteenth notes of beat 1, the third sixteenth note of beat 2, and the down beat of beat 3. The rhythmic tick initiated by the action of the left hand on the downbeat of beat 2, the third sixteenth note of beat 3, and the first and third sixteenth notes of beat 4 is produced by the action of the right hand stopping the strings as indicated in the tablature staff through the use of gray lines.  There are more than simply two reasons why an individual might choose to place her finger on a string. She could, for instance, want to ensure that a string will not vibrate sympathetically, thus producing overtones which may be undesirable to the performance of a composition. She may also want to ensure that the pitch of the string which is being stopped by the right hand does not accidentally sound while playing the guitar. This could be advantageous to the performer when she is playing in a tuning which does not complement the key of the compositions as is such in ``Ants'' by Leo Kottke. In the nylon-string world of guitarists such as Andrès Segovia, right-hand string-stopping is not notated. The harp is the only instrument that I have found which indicates when a string should stop sounding through the use of the damping symbol, \texttt{{\huge𝄌}}.
This composition requires a high degree of detailed, diligent practice in order to perform the melody correctly.

 \end{document}
%%% Local Variables:
%%% mode: latex
%%% TeX-master: t
%%% End:
